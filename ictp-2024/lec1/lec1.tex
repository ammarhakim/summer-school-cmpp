\documentclass[aspectratio=43]{beamer}
%\documentclass[aspectratio=169]{beamer}

\usepackage{graphicx}  % Required for including images
\usepackage{natbib}
\usepackage{booktabs} % Top and bottom rules for tables
\usepackage{amssymb,amsthm,amsmath}
\usepackage{exscale}
\usepackage{natbib}
\usepackage{tikz}
\usepackage{listings}
\usepackage{color}
\usepackage{bm}
\usepackage{hyperref}
\usepackage{animate}
\hypersetup{colorlinks=true}
\hypersetup{citecolor=porange}
\hypersetup{urlcolor=porange!80!}
\hypersetup{linkcolor=porange}

\newtheorem{proposition}{Proposition}
\newtheorem{remark}{Remark}
\newtheorem{principle}{Principle}

% Set the margins

%\newcommand{\grad}{\nabla}
\newcommand{\ignore}[1]{}  % ignore a section.

% Handy commands for notes and 'to do' items 

% For a clean copy, ignore these:
%\newcommand{\todo}[1]{}
%\newcommand{\note}[2]{}
%\newcommand\tocite[1]{}

%% Writing quarters
\newcommand{\wQ}[1]{{\textcolor{white}{Q#1}}}
\newcommand{\bQ}[1]{{Q#1}}

\DeclareMathAlphabet{\mathpzc}{OT1}{pzc}{m}{it}

%% Autoscaled figures
\newcommand{\incfig}{\centering\includegraphics}
\setkeys{Gin}{width=0.9\linewidth,keepaspectratio}

%% Commonly used macros
\newcommand{\eqr}[1]{Eq.\thinspace(#1)}
\newcommand{\pfrac}[2]{\frac{\partial #1}{\partial #2}}
% \newcommand{\pfracc}[2]{\frac{\partial^2 #1}{\partial #2^2}}
\newcommand{\pfraca}[1]{\frac{\partial}{\partial #1}}
\newcommand{\pfracb}[2]{\partial #1/\partial #2}
% \newcommand{\pfracbb}[2]{\partial^2 #1/\partial #2^2}
% \newcommand{\spfrac}[2]{{\partial_{#1}} {#2}}
% \newcommand{\mvec}[1]{\mathbf{#1}}
\newcommand{\gvec}[1]{\boldsymbol{#1}}
% \newcommand{\script}[1]{\mathpzc{#1}}
% \newcommand{\gke}{{\tt Gkeyll}}
\newcommand{\gcs}{\nabla_{\mvec{x}}}
\newcommand{\gvp}{\nabla_{\mvec{p}}}
\newcommand{\gvs}{\nabla_{\mvec{v}}}
\newcommand{\gps}{\nabla_{\mvec{z}}}
\newcommand{\dtx}{\thinspace d^3\mvec{x}}
\newcommand{\Perp}{{\mbox{$\scriptscriptstyle \perp$}}} 

\newcommand{\delx}{\nabla_\mvec{x}}
\newcommand{\delv}{\nabla_\mvec{v}}
\newcommand{\delxp}{\nabla_{\mvec{x}'}}
\newcommand{\delvp}{\nabla_{\mvec{v}'}}
\newcommand{\delxpp}{\nabla_{\mvec{x}''}}
\newcommand{\delvpp}{\nabla_{\mvec{v}''}}
\newcommand{\bdf}{\bar{f}}
\newcommand{\bdg}{\bar{g}}
\newcommand{\buni}{\mvec{b}}

%% NOAH stuff
\newcommand{\bhat}{\hat{\textsf{b}}}
\newcommand{\vparhat}{\hat{\textsf{v}}_\parallel}
\newcommand{\btilde}{\tilde{\textsf{b}}}
\newcommand{\vE}{{\bf v}_E}
\newcommand{\pderiv}[2]{
\frac{\partial #1}{\partial #2}
}
\newcommand{\balpha}{\boldsymbol{\alpha}}
\newcommand{\Ast}{{\scalebox{1.5}{\raisebox{-0.2ex}{$\ast$}}}}



%% MANA stuff

\newcommand{\LBO}{LBO}
\newcommand{\dx}{dx}
\newcommand{\dxi}{\Delta x_i}
\newcommand{\dxij}{\Delta x_{i,j}}
\newcommand{\dvi}{\Delta v_i}
\newcommand{\vt}{v_{th}}
\newcommand{\vts}{v_{th,s}}
\newcommand{\vte}{v_{th,e}}
\newcommand{\vti}{v_{th,i}}
\newcommand{\vv}{\v{v}}
\newcommand{\vi}{v_i}
\newcommand{\vj}{v_j}
\newcommand{\vk}{v_k}
\newcommand{\ui}{u_i}
\newcommand{\uk}{u_k}
\newcommand{\vu}{\v{u}}
\newcommand{\lapvi}[1]{\pd{}{\vi\partial\vi} #1}
\newcommand{\vX}{\v{x}}
\newcommand{\vZ}{\v{z}}
\newcommand{\vx}{v_x}
\newcommand{\vy}{v_y}
\newcommand{\vz}{v_z}
\newcommand{\lapv}[1]{\nabla^2_{\vv} #1}
\newcommand{\lapSqv}[1]{\nabla^4_{\vv} #1}
\newcommand{\divv}[1]{\nabla_{\vv} \cdot #1}
%\newcommand{\vE}{\v{E}}
\newcommand{\vB}{\v{B}}
\newcommand{\vJ}{\v{J}}
\newcommand{\Ji}{J_i}
\newcommand{\spr}{{s^\prime}}
\newcommand{\vpr}{{v^\prime}}
\newcommand{\vvpr}{\v{\vpr}}
\newcommand{\intInf}{\int_{-\infty}^{\infty}}
\newcommand{\dvv}{d\vv}
\newcommand{\dTvv}{d^2\vv}
\newcommand{\dThvv}{d^3\vv}
\newcommand{\ndim}{d}
\newcommand{\cdim}{d_c}
\newcommand{\vdim}{d_v}
\newcommand{\polyOrder}{p}
\newcommand{\bInd}{k}
\newcommand{\basisi}{{\psi_\bInd}}	% Basis function with index.
\newcommand{\polySet}{\v{P}^\polyOrder}
\newcommand{\dt}{{\Delta}t}
\newcommand{\vmax}{v_{\text{max}}}
\newcommand{\vmaxi}{v_{i,\text{max}}}
\newcommand{\vmin}{v_{\text{min}}}
\newcommand{\vmini}{v_{i,\text{min}}}
\newcommand{\vmaxj}{v_{j,\text{max}}}
\newcommand{\vminj}{v_{j,\text{min}}}
\newcommand{\vbound}{\partial\Omega_v}
\newcommand{\vboundi}{\partial\Omega_{v_i}}
\newcommand{\vSqPro}{\overline{v^2}}
\newcommand{\viSqPro}{\overline{v_i^2}}
\newcommand{\viviPro}{\overline{v_k v_k}}

\newcommand{\Ncells}{N}	% Number of cells.
\newcommand{\NC}{N_c} % Number of configuration space monomials.
\newcommand{\NV}{N_v} % Number of configuration space monomials.
\newcommand{\NP}{N_p} % Number of configuration space monomials.
\newcommand{\cellInd}{j}	% Letter used to index cells.
\newcommand{\cellLetter}{K} % Letter used to denote a cell.
\newcommand{\celli}{{\cellLetter_\cellInd}}	% Indexed cell.
\newcommand{\celliBound}{{\partial \celli}}	% Boundary of the cell.
\newcommand{\celliBoundvi}{{\partial K_{\cellInd,v_i}}}	% Boundary of the cell orthoginal to v_i.
\newcommand{\celliBoundvk}{{\partial K_{\cellInd,v_k}}}	% Boundary of the cell orthoginal to v_k.
\newcommand{\nhat}{\uv{n}}	% Unit normal vector.
\newcommand{\xijp}{{x_{i,\cellInd+1/2}}}	% Upper conf space boundary of the cell.
\newcommand{\xijm}{{x_{i,\cellInd-1/2}}} % Lower conf space boundary of the cell.
\newcommand{\vijp}{{v_{i,\cellInd+1/2}}}	% Upper velocity boundary of the cell.
\newcommand{\vijm}{{v_{i,\cellInd-1/2}}} % Lower velocity boundary of the cell.
\newcommand{\vkjp}{{v_{k,\cellInd+1/2}}}	% Upper velocity boundary of the cell.
\newcommand{\vkjm}{{v_{k,\cellInd-1/2}}} % Lower velocity boundary of the cell.

% Notation related to recovery DG.
\newcommand{\frec}{\hat{f}}	% Recovery polynomial.
\newcommand{\cellL}{\cellLetter_L}	% Left cell.
\newcommand{\cellR}{\cellLetter_R}	% Right cell.
\newcommand{\fhL}{f_{h L}}
\newcommand{\fhR}{f_{h R}}
\newcommand{\fhLi}{f_{h L,k}}
\newcommand{\fhRi}{f_{h R,k}}
\newcommand{\polySetL}{\polySet_L}
\newcommand{\polySetR}{\polySet_R}
\newcommand{\basisLi}{\psi_{L,k}}
\newcommand{\basisRi}{\psi_{R,k}}
\newcommand{\basisLj}{\psi_{L,l}}
\newcommand{\basisRj}{\psi_{R,l}}

% Moments.
\newcommand{\MZ}{M_0}
\newcommand{\MZs}{M_0^s}
\newcommand{\MO}{M_1}
\newcommand{\MOi}{M_{1,i}}
\newcommand{\MOis}{M_{1,i}^s}
\newcommand{\MOk}{M_{1,k}}
\newcommand{\MOs}{M_{1}^s}
\newcommand{\vMOs}{\gv{M}_{1}^s}
\newcommand{\vMO}{\gv{M}_1}
\newcommand{\MT}{M_2}
\newcommand{\MTs}{M_{2}^s}

\newcommand{\Ghat}{\hat{G}}
\newcommand{\fL}{f_L}
\newcommand{\fR}{f_R}
\newcommand{\fhat}{\hat{f}}

%--- Math commands ---
% \DeclareMathOperator{\Sample}{Sample}
\let\vaccent=\v % rename builtin command \v{} to \vaccent{}
\renewcommand{\v}[1]{\ensuremath{\mathbf{#1}}} % for vectors
\newcommand{\gv}[1]{\ensuremath{\mbox{\boldmath$ #1 $}}} 
% for vectors of Greek letters
\newcommand{\uv}[1]{\ensuremath{\mathbf{\hat{#1}}}} % for unit vector
\newcommand{\abs}[1]{\left| #1 \right|} % for absolute value

\newcommand{\avg}[1]{\left< #1 \right>} % for average
\let\underdot=\d % rename builtin command \d{} to \underdot{}
\renewcommand{\d}[2]{\frac{d #1}{d #2}} % for derivatives
\newcommand{\dd}[2]{\frac{d^2 #1}{d #2^2}} % for double derivatives
\newcommand{\pd}[2]{\frac{\partial #1}{\partial #2}} 
% for partial derivatives
\newcommand{\pdd}[2]{\frac{\partial^2 #1}{\partial #2^2}} 
% for double partial derivatives
\newcommand{\pddo}[3]{\frac{\partial^2 #1}{\partial #2 \partial #3}} 
% for off-diagonal double partial derivatives
\newcommand{\pdc}[3]{\left( \frac{\partial #1}{\partial #2}\right)_{#3}} % for thermodynamic partial derivatives
\newcommand{\partialx}[1]{\partial_x #1} % partial_x (simplified notation for \pd)
\newcommand{\partialy}[1]{\partial_y #1} % partial_y (simplified notation for \pd)
\newcommand{\partialz}[1]{\partial_z #1} % partial_z (simplified notation for \pd)
%\newcommand{\ket}[1]{\left| #1 \right>} % for Dirac bras
%\newcommand{\bra}[1]{\left< #1 \right|} % for Dirac kets
%\newcommand{\braket}[2]{\left< #1 \vphantom{#2} \right|
% \left. #2 \vphantom{#1} \right>} % for Dirac brackets
%\newcommand{\matrixel}[3]{\left< #1 \vphantom{#2#3} \right|
% #2 \left| #3 \vphantom{#1#2} \right>} % for Dirac matrix elements
\newcommand{\grad}[1]{\nabla #1} % for gradient
\let\divsymb=\div % rename builtin command \div to \divsymb
\renewcommand{\div}[1]{\nabla \cdot #1} % for divergence
\newcommand{\curl}[1]{\nabla \times #1} % for curl
\newcommand{\lap}[1]{\nabla^2 #1} % Laplacian.
\newcommand{\inProd}[2]{{\left\langle #1, #2 \right\rangle}} % inner product.
\newcommand{\kron}[2]{\delta_{#1 #2}} % Kronecker delta.
\newcommand{\coll}[1]{\left(\pd{ #1}{t}\right)_c} % Collision operator.
\newcommand{\collF}[1]{\left(\partial{ #1}/\partial{t}\right)_c} %

% Other stuff
%\usepackage{units}
\usepackage{graphicx}
\usepackage{color}
\usepackage{hyperref}

\usepackage{tabularx}
\usepackage{mdwlist} % enumerate*, itemize* squeeze vertical space

\newcommand{\comment}[1]{\textit{\textcolor{red}{#1}}}
\renewcommand{\comment}[1]{}

% Turn off red comments by uncommenting the following line:
%\renewcommand{\comment}[1]{}

%% Autoscaled figures
%\newcommand{\incfig}{\centering\includegraphics}
\setkeys{Gin}{width=0.9\linewidth,keepaspectratio}
\newcommand{\gke}{{\tt Gkeyll}}
%\newcommand{\gke}{{\textsc{Gkeyll}}}
\newcommand{\mvec}[1]{\mathbf{#1}}

%Make the items smaller
\newcommand{\cramplist}{
	\setlength{\itemsep}{0in}
	\setlength{\partopsep}{0in}
	\setlength{\topsep}{0in}}
\newcommand{\cramp}{\setlength{\parskip}{.5\parskip}}
\newcommand{\zapspace}{\topsep=0pt\partopsep=0pt\itemsep=0pt\parskip=0pt}

\newcommand{\backupbegin}{
   \newcounter{finalframe}
   \setcounter{finalframe}{\value{framenumber}}
}
\newcommand{\backupend}{
   \setcounter{framenumber}{\value{finalframe}}
 }

\newcommand{\mypause}{\pause}
%\newcommand{\mypause}{}

\usetheme[bullet=circle,% Use circles instead of squares for bullets.
titleline=true,% Show a line below the frame title.
]{Princeton}

\title[{\tt }] {Computational Methods for Fluids \& (Gyro) Kinetic
  Equations. Lecture I}%
\author[http://cmpp.rtfd.io]%
{Ammar H. Hakim\inst{1}}%

\institute[PPPL]
{ \inst{1} Princeton Plasma Physics Laboratory, Princeton, NJ %
}

\date[5/9/2024]{ICTP Fusion Summer School, 2024}

\begin{document}

\begin{frame}
  \titlepage
\end{frame}

\begin{frame}

  \begin{block}{Alternative Title}
    A Field Guide to Constructing Discrete Plasma Universes
  \end{block}  

\end{frame}

\begin{frame}{Computational Plasma Physics: Uniquely Challenging}
  \footnotesize Vast majority of plasma physics in contained in the
  Vlasov-Maxwell equations that describe self-consistent evolution of
  distribution function $f(\mvec{x},\mvec{v},t)$ and electromagnetic
  fields:
  \begin{align*}
    \pfrac{f_s}{t} + \gcs\cdot (\mvec{v} f_s) + \gvs\cdot (\mvec{F}_s
    f_s) = \left( \pfrac{f_s}{t} \right)_c
  \end{align*}
  where $\mvec{F}_s=q_s/m_s (\mvec{E}+\mvec{v}\times\mvec{B})$. The EM
  fields are determined from Maxwell equations
  \begin{align*}
    \frac{\partial \mvec{B}}{\partial t} + \nabla\times\mvec{E} &= 0 \\
    \epsilon_0\mu_0\frac{\partial \mvec{E}}{\partial t} -
    \nabla\times\mvec{B} &= -\mu_0\sum_s q_s \int_{-\infty}^\infty v f_s \thinspace d\mvec{v}^3
  \end{align*}
  Theoretical and computational plasma physics consists of making
  extensions/approximations and solving these equations in specific
  situations.
\end{frame}

\begin{frame}{Why is solving Vlasov-Maxwell equations hard?}
  
  Despite being the fundamental equation in plasma physics the VM
  equations remain highly challenging to solve.
  \begin{itemize}
  \item Highly nonlinear with the coupling between fields and
    particles via currents and Lorentz force. Collisions can further
    complicate things
    \mypause%
  \item High dimensionality and multiple species with large mass
    ratios: 6D phase-space, $m_e/m_p = 1/1836$ and possibly dozens of
    species.
    \mypause%
  \item Enormous scales in the system: light speed and electron plasma
    oscillations; cyclotron motion of electrons and ions; fluid-like
    evolution on intermediate scales; resistive slow evolution of
    near-equilibrium states; transport scale evolution in tokamak
    discharges.  14 orders of magnitude of physics in these
    equations!
  \end{itemize}
\end{frame}

\begin{frame}{Goals and Outline of Talk}
  Goal of these lectures is to introduce modern concepts in
  computational plasma physics, specially with a view towards
  connecting \emph{continuous} and \emph{discrete} properties of the
  equations.
  
  \mypause%
  \begin{itemize}
  \item {\bf Part 1}: Physics that should be preserved in the discrete
    system. Indirect properties and going beyond accuracy and order of
    schemes.
    \mypause%
  \item {\bf Part 2:} Schemes for fluid and (gyro) kinetic
    equations. Mostly focussed on Finite-Volume (FV) schemes for
    multi-fluid equations, and advanced discontinuous Galerkin (DG)
    for (gyro) kinetic equations.
  \end{itemize}
  \vskip0.1in%
  Treat kinetic and multifluid, multimoment equations as PDEs. Not a
  talk on PIC methods! (See work by H. Qin, P. Morrison et al on
  modern structure preserving schemes).
\end{frame}

\begin{frame}{Why Care?}

  One can look at computational physics in two ways: as an end in
  itself, and as a tool for applications. Both of these are important!

  \mypause%  
  As end in itself:
  \begin{itemize}
  \item Sits between applied mathematics and theoretical physics. The
    goal is to design efficient numerical methods to solve equations
    from theoretical physics.
    \mypause%
  \item The goal here is the numerical method itself: what are its
    properties? Does it faithfully represent the underlying physics?
    Does it run efficiently on modern computers? Research into modern
    numerical methods (including structure preserving methods) fall
    into this category.
    \mypause%    
  \item Usually, besides the fun of solving complex equations, we wish
    to gain deeper understanding of underlying physics. {\bf Some
      theoretical questions can only be answered with computer
      simulations.}
  \end{itemize}
  
\end{frame}

\begin{frame}{Why Care?}

  As a tool for applications:
  \begin{itemize}
  \item The second reason to care is that computational physics
    provides tools to understand/design experiments or observations.
  \item {Large number of routine calculations are needed to build
      modern experiments (heat-transfer, structural analysis, basic
      fluid mechanics, equilibrium and stability calculations,
      etc). {\bf Such routine calculations are no longer cutting edge
        research topics.}}
  \end{itemize}
  \mypause%  
  \vskip0.1in%
  However, today strong need to be at {\bf intersection of
    cutting-edge computational physics and critical applications}:
  E.g: More than \$6 billion are invested in private fusion efforts;
  billions more in public efforts.
\end{frame}

\begin{frame}{Exploring the Fusion Design Space}

  What is needed to explore or ``confine'' the design space for the
  crowded space of fusion concepts?

  \begin{itemize}
  \item Unfortunately, neither the physical models or the numerics are
    fully developed yet to understand \emph{burning plasma
      regime}. Enormous scales, hairy plasma-material-interaction and
    zoo of possible instabilities.
  \end{itemize}
  \mypause%
  Two approaches: 
  \begin{itemize}  
  \item Do calculations with first-principles models on the large
    computers. Required to do detailed physics studies%
    \mypause%
  \item Do many smaller calculations in an \emph{optimization loop} to
    confine the design space. Run occasional large calculations to
    verify.
  \end{itemize}

  Each of these approaches needs development of appropriate models,
  and fast numerical methods on modern hardware architecture. Put
  everything in an optimization loop.

\end{frame}

\begin{frame}{Setting the Stage: Shock-Bubble Interactions}
  \begin{figure}
    %\setkeys{Gin}{width=0.9\linewidth,keepaspectratio}
    %\incfig{shock-bubble/rho_shock_bubble_40.png}
    \animategraphics[controls={play,step,stop},width=0.7\linewidth]{10}{shock-bubble/rho_shock_bubble_}{0}{40}
  \end{figure}
\end{frame}

\begin{frame}{Setting the Stage: Shock-Bubble Interactions}
  \begin{figure}
    \setkeys{Gin}{width=0.7\linewidth,keepaspectratio}
    \incfig{shock-bubble/rho_shock_bubble_40.png}
    %\animategraphics[controls={play,step,stop},width=0.65\linewidth]{10}{shock-bubble/rho_shock_bubble_}{0}{40}
  \end{figure}
\end{frame}

\begin{frame}{Setting the Stage: Kevin-Helmholtz Instability}
  \begin{figure}
    %\setkeys{Gin}{width=0.65\linewidth,keepaspectratio}
    %\incfig{kh/kh-hr_10.png}
    \animategraphics[controls={play,step,stop},width=0.65\linewidth]{10}{kh/kh-hr_}{0}{40}
  \end{figure}
\end{frame}

\begin{frame}{Setting the Stage: Kevin-Helmholtz Instability}
  \begin{figure}
    \setkeys{Gin}{width=0.65\linewidth,keepaspectratio}
    \incfig{kh/kh-hr_10.png}
    %\animategraphics[controls={play,step,stop},width=0.65\linewidth]{10}{kh/kh-hr_}{0}{40}
  \end{figure}
\end{frame}


\begin{frame}{The Ultimate Discrete Scheme (UDS)}

  A hypothetical ``Ultimate Discrete Scheme'' must possesses the
  following three properties
  \begin{itemize}
  \item {\bf Robustness} The scheme must be \emph{robust}: capture
    shocks, maintain positivity, preserve monotonicity, satisfy
    involutions (divergence constraints), properly preserve energy
    partition.
  \item {\bf Accuracy} Provide low dissipation for smooth high-$k$
    modes to properly simulate turbulence. Converge quickly to give
    accurate results when needed
  \item {\bf Efficiency} Run rapidly for modest resolutions. Do
    interesting physics on a laptop. Use GPUs and other hardware
    accelerators for larger simulations. Do 1000s of simulations.
  \end{itemize}

  Sadly, such a scheme does not exist! Many of the goals are
  contradictory.
\end{frame}

\begin{frame}{No Free Lunch Principle}

  \begin{block}{No Free Lunch Principle}
    There is no \emph{unique} discrete system of equations
    corresponding to a given system of continuous equations. No
    discrete system is perfect and a method that works well in one
    situation may not work well in others.
  \end{block}

  \mypause%
  ``All numerical methods suck, though some suck less than
  others. Make sure your method sucks less that the competition''

\end{frame}

\begin{frame}{Many approximations developed over the decades}
  \small%
  Modern computational plasma physics consists of making justified
  approximations to the VM system and then coming up with efficient
  schemes to solve them.
  \mypause%
  \begin{itemize}
  \item Major recent theoretical development in plasma physics is the
    discovery of gyrokinetic equations, an asymptotic approximation
    for plasmas in strong magnetic fields. Reduces dimensionality to
    5D (from 6D) and eliminates cyclotron frequency and gyroradius
    from the system. Very active area of research.
\mypause%
  \item Many fluid approximations have been developed to treat plasma
    via low-order moments: extended MHD models; multimoment models;
    various reduced MHD equations
    \mypause%
  \item Numerical methods for these equations have undergone
    renaissance in recent years: emphasis on \emph{memetic} schemes
    that preserve conservation laws and some geometric features of the
    continuous equations. Based on Lagrangian and Hamiltonian
    formulation of basic equations. Very active area of research.
  \end{itemize}
  With advent of large scale computing much research is now focused on
  schemes that scale well on thousands (millions) of CPU/GPU cores.
\end{frame}

\begin{frame}{Integrating kinetic effects in fluid models}

  \begin{itemize}
  \item For physically accurate simulations of various fusion
    machines, its important to {\bf go beyond resistive and Hall-MHD}.
  \item Traditional approach has been to use a generalized Ohm's law,
    adding physics to it in a piecemeal fashion.
  \item However, this approach has limited success, and in particular,
    there is no systematic way to add important collisionless kinetic
    effects in a self-consistent and numerically tractable manner.
  \item A major challenge in the fusion and other applications is that
    the {\bf plasma is nearly collisionless}, and that the {\bf
      magnetic fields (external coils, planetary dipole) add a
      preferred direction}, adding significant anisotropy to the
    system.
  \end{itemize}

\end{frame}

\begin{frame}{Alternative is to use
    multi-fluid moment models}

  \begin{itemize}
  \item In this approach we take moments of the Vlasov equation,
    truncating the moment sequence using a closure.
  \item The interaction between species is via electromagnetic fields,
    which are evolved using Maxwell equations (retaining displacement
    currents)
  \item This approach allows natural and self-consistent inclusion of
    {\bf finite electron inertia, Hall currents, anisotropic pressure
      tensor and heat flux tensor}.
  \item Even though the multi-fluid moment equations contain physics
    all the way from light waves and electron dynamics to MHD scales,
    by use of advanced algorithms very efficient and robust schemes
    can be developed, {\bf allowing us to treat a sequence of
      increasing fidelity models in a uniform and consistent manner}.
  \end{itemize}

\end{frame}

\begin{frame}{Sequence of models with 5, 10 and 20 moments}
  \small%
  Taking moments of Vlasov equation leads to the \emph{exact} moment
  equations listed below
  \begin{align*}
    \pfrac{n}{t}+\pfraca{x_j}(nu_j) &= 0 \\
    m\pfraca{t}(nu_i) + \pfrac{\mathcal{P}_{ij}}{x_j} &= nq(E_i+\epsilon_{ijk}u_jB_k) \\
    \pfrac{\mathcal{P}_{ij}}{t} + \pfrac{\mathcal{Q}_{ijk}}{x_k} &= nqu_{[i}E_{j]} + \frac{q}{m}\epsilon_{[ikl}\mathcal{P}_{kj]}B_l\\
    \pfrac{\mathcal{Q}_{ijk}}{t} + \pfrac{\mathcal{K}_{ijkl}}{x_l} &= \frac{q}{m}(E_{[i}\mathcal{P}_{jk]} + \epsilon_{[ilm}\mathcal{Q}_{ljk]}B_m)
  \end{align*}
  In the {\bf five-moment} model, we assume that the pressure is
  isotropic $P_{ij} = p \delta_{ij}$. For the {\bf ten-moment} model,
  we include the time-dependent equations for all six components of
  the pressure tensor, and use a closure for the heat-flux. In the
  {\bf twenty-moment} model, we evolve all ten components of the
  heat-flux tensor, closing at the fourth moment.

\end{frame}

\begin{frame}{Planetary Scale Simulations Are Now Possible!}

  \begin{figure}
    \setkeys{Gin}{width=0.75\linewidth,keepaspectratio}
    \incfig{Mercury-1.png}
    \caption{Ten-Moment Simulation of Mercury's Magnetosphere (Dong
      et. al. GRL, {\bf 46}, 2019}
  \end{figure}
  
\end{frame}

\begin{frame}{Reconnection On Mercury's Night and Day Sides}

  \begin{figure}
    \setkeys{Gin}{width=0.9\linewidth,keepaspectratio}
    \incfig{Mercury-Reconnection.png}
  \end{figure} 
  
\end{frame}

\begin{frame}{Earth-Scale Simulations Are ``Almost'' Possible}

  \begin{figure}
    \setkeys{Gin}{width=0.95\linewidth,keepaspectratio}
    \incfig{Earth-Magnetosphere.png}
    \caption{Ten-Moment simulation of Earth's magnetosphere shows
      extended night-side current sheet, just after a disruption
      driven by ballooning instability. Wang et. al. JCP {\bf 415},
      2020}
  \end{figure} 
  
\end{frame}

\begin{frame}{More Planetary Magnetospheres: Uranus}

  \begin{figure}
    \setkeys{Gin}{width=0.75\linewidth,keepaspectratio}
    \incfig{Uranus-Magnetosphere.png}
    \caption{Ten-Moment simulation of Uranus's magnetosphere shows
      extremely complex magnetic-field structure due to dipole axis
      tilt with respective to revolution plane.}
  \end{figure}  

\end{frame}

\begin{frame}{Simple harmonic oscillator}
  \small%
  Consider first the simple harmonic oscillator
  \begin{align*}
    \frac{d^2z}{dt^2} = -\omega^2 z
  \end{align*}
  This has exact solution $z = a\cos(\omega t) + b\sin(\omega t)$,
  where $a$ and $b$ are arbitrary constants. How to solve this
  numerically? Write as a system of first-order ODEs
  \begin{align*}
    \frac{dz}{dt} = v; \quad \frac{dv}{dt} = -\omega^2 z
  \end{align*}
  Note that the coordinates $(z,v)$ label the \emph{phase-space} of
  the harmonic oscillator. Multiply the second equation by $v$ and use
  the first equation to get
  \begin{align*}
    \frac{d}{dt}\left(\frac{1}{2} v^2 + \frac{1}{2}\omega^2 z^2\right) = 0.
  \end{align*}
  This is the \emph{energy} and is \emph{conserved}.%
  
  \bf{Question: how to solve the ODE such that the energy is conserved
    by the \emph{discrete scheme}?}
\end{frame}

\begin{frame}{Harmonic oscillator: Forward Euler Scheme}
  First attempt: use the simplest possible scheme, replace derivatives
  with difference approximations
  \begin{align*}
    \frac{z^{n+1}-z^n}{\Delta t} = v^n; \quad \frac{v^{n+1}-v^n}{\Delta t} = -\omega^2 z^n
  \end{align*}
  or
  \begin{align*}
    z^{n+1} = z^n + \Delta t v^n; \quad v^{n+1} = v^n - \Delta t \omega^2 z^n
  \end{align*}
  This is the \emph{forward Euler} scheme. Lets check if the discrete
  scheme conserves energy:
  \begin{align*}
    (v^{n+1})^2 + \omega^2 (z^{n+1})^2 =
    (1+\omega^2 \Delta t^2)((v^{n})^2 + \omega^2 (z^{n})^2)
  \end{align*}
  The presence of the $\omega^2 \Delta t^2$ in the bracket spoils the
  conservation. So the forward Euler scheme \emph{does not} conserve
  energy. Also, note that the energy, in fact, is \emph{increasing}!
\end{frame}

\begin{frame}{Harmonic oscillator: Forward Euler Scheme}
  Closer look: write as a matrix equation
  \begin{align*}
    \left[
    \begin{matrix}
      z^{n+1} \\
      v^{n+1}
    \end{matrix}
    \right]
    =
    \underbrace{
    \left[
    \begin{matrix}
      1 & \Delta t \\
      -\omega^2 \Delta t & 1
    \end{matrix}
    \right]
    }_{\textrm{Jacobian},\ J}
    \left[
    \begin{matrix}
      z^{n} \\
      v^{n}
    \end{matrix}
    \right].                        
  \end{align*}
  Observe that the determinant of the Jacobian is
  $\det(J)=(1+\omega^2\Delta t^2)$ which is the same factor as appears
  in the energy relation. One may reasonably conjecture that when this
  determinant is one, then perhaps energy is conserved.

  \begin{block}{Volume Preserving Scheme}
    We will call say a scheme preserves \emph{phase-space} volume
    if the determinant of the Jacobian is $\det(J)=1$.
  \end{block}

\end{frame}

\begin{frame}{Harmonic oscillator: Mid-point Scheme}
  \small%
  Perhaps a better approximation will be obtained if we use
  \emph{averaged} values of $z,v$ on the RHS of the discrete
  equation:
  \begin{align*}
    \frac{z^{n+1}-z^n}{\Delta t} &= \frac{v^n +v^{n+1}}{2} \\
    \frac{v^{n+1}-v^n}{\Delta t} &= -\omega^2\frac{z^n + z^{n+1}}{2}
  \end{align*}
  This is an \emph{implicit} method as the solution at the next
  time-step depends on the old as well as the next time-step
  values. In this simple case we can explicitly write the update in a
  matrix form as
  \begin{align*}
    \left[
    \begin{matrix}
      z^{n+1} \\
      v^{n+1}
    \end{matrix}
      \right]
      =
      \frac{1}{1+\omega^2\Delta t^2/4}
    \left[
      \begin{matrix}
        1-\omega^2\Delta t^2/4  & \Delta t \\
        -\omega^2 \Delta t & 1-\omega^2\Delta t^2/4
      \end{matrix}
    \right]
    \left[
    \begin{matrix}
      z^{n} \\
      v^{n}
    \end{matrix}
    \right].                        
  \end{align*}
  For this scheme $\det(J)=1$. So the mid-point scheme conserves
  phase-space volume! Some algebra also shows that
  \begin{align*}
    (v^{n+1})^2 + \omega^2 (z^{n+1})^2 = (v^{n})^2 + \omega^2 (z^{n})^2
  \end{align*}
  showing that energy is also conserved by the mid-point scheme.
\end{frame}

\begin{frame}{Harmonic oscillator: Mid-point Scheme is symplectic}
  A more stringent constraint on a scheme for the simple harmonic
  oscillator is that it be \emph{symplectic}. To check if a scheme is
  symplectic one checks to see if
  \begin{align*}
    J^T\sigma J = \sigma
  \end{align*}
  where $\sigma$ is the \emph{unit symplectic matrix}
  \begin{align*}
    \sigma
    =
    \left[
      \begin{matrix}
        0 & 1 \\
        -1 & 0
      \end{matrix}
    \right]    
  \end{align*}
  Turns out that the mid-point scheme for the harmonic oscillator is
  also symplectic. Note that if a scheme conserves phase-space
  volume, it \emph{need not} be symplectic.
\end{frame}

\begin{frame}{Accuracy and Stability}
  \small%
  To study the stability, accuracy and convergence of a scheme one usually
  looks at the first order ODE
  \begin{align*}
    \frac{dz}{dt} = -\gamma z
  \end{align*}
  where $\gamma = \lambda + i\omega$ is the complex frequency. The
  exact solution to this equation is $z(t) = z_0 e^{-\gamma t}$. The
  solution has damped/growing modes ($\lambda>0$ or $\lambda<0$) as
  well as oscillating modes.

  \begin{itemize}
    \item The forward Euler scheme for this equation is
      \begin{align*}
        z^{n+1} = z^n - \Delta t \gamma z^n = (1-\Delta t\gamma) z^n.
      \end{align*}
    \item The mid-point scheme for this equation is
      \begin{align*}
        z^{n+1} = \left(\frac{1-\gamma\Delta t/2}{1+\gamma\Delta t/2}\right) z^n
      \end{align*}
    \end{itemize}
\end{frame}

\begin{frame}{Accuracy and Stability}
  We can determine how \emph{accurate} the scheme is by looking at how
  many terms the scheme matches the Taylor series expansion of the
  exact solution:
  \begin{align*}
    z(t^{n+1}) = z(t^n)
    \left(
    1 - \gamma\Delta t + \frac{1}{2}\gamma^2 \Delta t^2 - \frac{1}{6}\gamma^3 \Delta t^3 + \ldots
    \right)
  \end{align*}

  \begin{itemize}
  \item The forward Euler scheme matches the \emph{first two terms}
    \begin{align*}
      z^{n+1} = z^n (1-\Delta t\gamma)
    \end{align*}
  \item The mid-point scheme matches the \emph{first three terms}
    \begin{align*}
      z^{n+1} = z^n \left(
      1-\Delta t\gamma - \frac{1}{2}\gamma^2 \Delta t^2  - \frac{1}{4} \gamma^3 \Delta t^3 + \ldots
      \right)
    \end{align*}
  \end{itemize}
\end{frame}

\begin{frame}{Accuracy and Stability}
  We can determine if the scheme is \emph{stable} by looking at the
  amplification factor $| z^{n+1}/z^n|$. Note that for damped modes
  ($\lambda>0$) this quantity \emph{decays} in time, while for purely
  oscillating modes ($\lambda = 0$) this quantity remains
  \emph{constant}.

  
  \begin{itemize}
  \item The amplification factor for the forward Euler scheme in the
    absence of damping is $1+\omega^2\Delta t^2 > 1$, hence this
    scheme is \emph{unconditionally unstable}.
  \item The amplification factor for the mid-point scheme in the
    absence of damping is exactly 1, showing that the mid-point scheme
    is \emph{unconditionally stable}, that is, one can take as large
    time-step one wants without the scheme ``blowing up''. Of course,
    the errors will increase with larger $\Delta t$.
  \end{itemize}
\end{frame}

\begin{frame}{Runge-Kutta schemes}
  \begin{itemize}
  \item Even though the forward Euler scheme is unconditionally
    unstable, we can use it to construct other schemes that \emph{are}
    stable and are also more accurate (than first order).
  \item For example, a class of Runge-Kutta schemes can be written as
    a combination of forward Euler updates. In particular, the
    \emph{strong stability preserving} schemes are important when
    solving hyperbolic equations. Note that these RK schemes will
    \emph{not} conserve energy for the harmonic oscillator, but
    \emph{decay} it.
  \item Other multi-stage Runge-Kutta schemes can be constructed that
    allow very large time-steps for diffusive processes, for example,
    that come about when time-stepping diffusion equations.
  \end{itemize}
\end{frame}

\begin{frame}{Simple harmonic oscillator}
  We looked at
  \begin{align*}
    \frac{d^2z}{dt^2} = -\omega^2 z
  \end{align*}
  and wrote it as system of first-order ODEs
  \begin{align*}
    \frac{dz}{dt} = v; \quad \frac{dv}{dt} = -\omega^2 z
  \end{align*}
  Now introduce energy-angle coordinates
  \begin{align*}
    \omega z = E\sin\theta; \quad v = E\cos\theta
  \end{align*}
  then $E^2 = \omega^2 z^2 + v^2 \equiv E_0^2$ is a constant as we
  showed before. Using these expressions we get the very simple ODE
  $\dot{\theta} = \omega$. This shows that in phase-space
  $(v,\omega z)$ the motion is with uniform angular speed along a
  circle.
\end{frame}

\begin{frame}{Simple harmonic oscillator: Phase-errors}
  \small%
  The mid-point scheme had
  \begin{align*}
    (v^{n+1})^2 + \omega^2 (z^{n+1})^2 = (v^{n})^2 + \omega^2 (z^{n})^2
    = E_0^2
  \end{align*}
  which means that the mid-point scheme gets the energy coordinate
  \emph{exactly} correct. However, we have
  \begin{align*}
    \tan\theta^{n+1} = \frac{\omega z^{n+1}}{v^{n+1}}.
  \end{align*}
  Using the expressions for the scheme and Taylor expanding in
  $\Delta t$ we get
  \begin{align*}
    \tan\theta^{n+1}
    =
    \tan\theta^n +
    \frac{\omega E_0^2}{(v^{n})^2}\Delta t +
    \frac{\omega^3 z^n E_0^2}{(v^n)^3}\Delta t^2
    + O(\Delta t^3)
  \end{align*}
  The first three terms match the Taylor expansion of the exact
  solution $\tan(\theta^n+\omega\Delta t)$ and the last term is the
  \emph{phase-error}.
\end{frame}

\end{document}

\begin{frame}{}
\end{frame}

% ----------------------------------------------------------------
