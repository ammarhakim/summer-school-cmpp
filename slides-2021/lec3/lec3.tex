\documentclass[aspectratio=169]{beamer}
%\documentclass[aspectratio=43]{beamer}

\usepackage{graphicx}  % Required for including images
\usepackage{natbib}
\usepackage{booktabs} % Top and bottom rules for tables
\usepackage{amssymb,amsthm,amsmath}
\usepackage{exscale}
\usepackage{natbib}
\usepackage{tikz}
\usepackage{listings}
\usepackage{color}
\usepackage{animate}
\usepackage{bm}
\usepackage{etoolbox}

% Setup TikZ
\usepackage{tikz}
\usetikzlibrary{arrows}
\tikzstyle{block}=[draw opacity=0.7,line width=1.4cm]
% Setup hyperref
\usepackage{hyperref}
\hypersetup{colorlinks=true}
\hypersetup{citecolor=porange}
\hypersetup{urlcolor=porange!80!}
\hypersetup{linkcolor=porange}

%% Writing quarters
\newcommand{\wQ}[1]{{\textcolor{white}{Q#1}}}
\newcommand{\bQ}[1]{{Q#1}}

% Uncomment appropriate command to disable/enable hiding
%\newcommand{\mypause}{\pause}
\newcommand{\mypause}{}
\newcommand{\myb}[1]{{\color{blue} {#1}}}

%% Commonly used macros
\newcommand{\eqr}[1]{Eq.\thinspace(#1)}
\newcommand{\pfrac}[2]{\frac{\partial #1}{\partial #2}}
\newcommand{\pfracc}[2]{\frac{\partial^2 #1}{\partial #2^2}}
\newcommand{\pfraca}[1]{\frac{\partial}{\partial #1}}
\newcommand{\pfracb}[2]{\partial #1/\partial #2}
\newcommand{\pfracbb}[2]{\partial^2 #1/\partial #2^2}
\newcommand{\spfrac}[2]{{\partial_{#1}} {#2}}
\newcommand{\mvec}[1]{\mathbf{#1}}
\newcommand{\gvec}[1]{\boldsymbol{#1}}
\newcommand{\script}[1]{\mathpzc{#1}}
\newcommand{\eep}{\mvec{e}_\phi}
\newcommand{\eer}{\mvec{e}_r}
\newcommand{\eez}{\mvec{e}_z}
\newcommand{\iprod}[2]{\langle{#1}\rangle_{#2}}

\newcommand{\gcs}{\nabla_{\mvec{x}}}
\newcommand{\gvs}{\nabla_{\mvec{v}}}
\newcommand{\gps}{\nabla_{\mvec{z}}}
\newcommand{\dtv}{\thinspace d^3\mvec{v}}
\newcommand{\dtx}{\thinspace d^3\mvec{x}}

\newtheorem{thm}{Theorem}
\newtheorem{lem}{Lemma}
\newtheorem{remark}{Remark}

\newtheorem{proposition}{Proposition}

\theoremstyle{definition}
\newtheorem{dfn}{Definition}
%\newtheorem{thm}{Theorem}
\newtheorem{proposed}{Proposed Answer}
\DeclareMathOperator{\spn}{span}

\DeclareMathAlphabet{\mathpzc}{OT1}{pzc}{m}{it}

%% Autoscaled figures
\newcommand{\incfig}{\centering\includegraphics}
\setkeys{Gin}{width=0.9\linewidth,keepaspectratio}

%Make the items smaller
\newcommand{\cramplist}{
	\setlength{\itemsep}{0in}
	\setlength{\partopsep}{0in}
	\setlength{\topsep}{0in}}
\newcommand{\cramp}{\setlength{\parskip}{.5\parskip}}
\newcommand{\zapspace}{\topsep=0pt\partopsep=0pt\itemsep=0pt\parskip=0pt}

\newcommand{\backupbegin}{
   \newcounter{finalframe}
   \setcounter{finalframe}{\value{framenumber}}
}
\newcommand{\backupend}{
   \setcounter{framenumber}{\value{finalframe}}
}

\usetheme[bullet=circle,% Use circles instead of squares for bullets.
          titleline=true,% Show a line below the frame title.
          ]{Princeton}

\title[{\tt }]{Solving Maxwell Equations and Intro to Hyperbolic PDEs}%
\author[http://cmpp.rtfd.io]%
{Ammar H. Hakim ({\tt ammar@princeton.edu}) \inst{1}}%

\institute[PPPL]
{ \inst{1} Princeton Plasma Physics Laboratory, Princeton, NJ %
}

\date[8/18/2021]{PPPL Graduate Summer School, 2021}

\begin{document}

\begin{frame}[plain]
  \titlepage
\end{frame}

\begin{frame}{Solving Maxwell equations}
  \small%
  Besides pushing particles in electromagnetic fields, we need to
  compute these fields self-consistently from currents and charges by
  solving Maxwell equations. First consider Maxwell equations in
  vacuum:
  \begin{align*}
    \frac{\partial \mvec{B}}{\partial t} + \nabla\times\mvec{E} &= 0 \\
    \epsilon_0\mu_0\frac{\partial \mvec{E}}{\partial t} -
    \nabla\times\mvec{B} &= 0
  \end{align*}
  For these we have the conservation laws
  \begin{align*}
    \frac{d}{dt} \int_\Omega \mvec{E}\times\mvec{B}  \dtx &= 0 \\
    \frac{d}{dt} \int_\Omega \left( \frac{\epsilon_0}{2} |\mvec{E}|^2 + \frac{1}{2\mu_0} |\mvec{B}|^2 \right) \dtx &= 0.
  \end{align*}
  Note these are \emph{global} conservation laws and one can instead
  also write \emph{local} conservation laws that include momentum and
  energy flux terms. How to solve these equations efficiently and
  maintain (some) conservation and geometric properties?
\end{frame}

\begin{frame}{Solving Maxwell equations}
  \begin{itemize}
  \item Maxwell equations have a very special geometric structure. The
    electric field $\mvec{E}$ is a \emph{vector} while the magnetic
    field $\mvec{B}$ is a \emph{bivector} (this is disguised in the
    usual formulations of Maxwell equations).
  \item (In spacetime formulations the complete electromagnetic field
    is represented as a single bivector in 4D space-time).
  \item The fact that we are dealing with two objects of
    \emph{different} geometric types indicates that the discrete
    Maxwell equations should also inherit this somehow.
  \item The Yee algorithm, often called the \emph{finite-difference
      time-domain} algorithm is the most successful (and simple)
    algorithm that accounts of this geometric structure. It is
    implemented in most PIC codes, though recent research has focused
    on structure preserving finite-element and other methods.
  \end{itemize}
\end{frame}

\begin{frame}{Solving Maxwell equations: The Yee-cell}
  \begin{figure}
    \setkeys{Gin}{width=0.5\linewidth,keepaspectratio}
    \incfig{Yee-cube.png}
    \caption{Standard Yee-cell. Electric field components (vectors)
      are located on edges while magnetic field components
      (bivectors) are located on faces.}
  \end{figure}  
\end{frame}

\begin{frame}{Solving Maxwell equations: The Yee-cell}
  \small%
  On the Yee-cell the difference approximation to Maxwell equations
  ``falls out'', almost like magic. The updates are staggered in time
  and use two \emph{different} discrete curl operators:
  \begin{align*}
    \mvec{B}^{n+1/2} &= \mvec{B}^{n-1/2} - \Delta t\thinspace \nabla_E\times\mvec{E}^{n} \\    
    \mvec{E}^{n+1} &= \mvec{E}^n + \Delta t/c^2\thinspace \nabla_F\times\mvec{B}^{n+1/2}
  \end{align*}
  Here the symbols $\nabla_F\times$ and $\nabla_E\times$ are the
  discrete curl operators:
  \begin{itemize}
  \item The first takes \emph{face-centered} magnetic field and
    computes it curl. This operator \emph{puts the result on cell
      edges}.
  \item The second takes \emph{edge-centered} electric field and
    computes it curl. This operator \emph{puts the result on cell
      faces}.
  \item The structure of Yell-cell also indicates that \emph{currents}
    must be co-located with the electric field and computed at half
    time-steps.
  \end{itemize}
  This duality neatly reflects the underlying geometry of Maxwell
  equations. The staggering in time reflects the fact that in 4D the
  electromagnetic field is a bivector in spacetime.
\end{frame}

\begin{frame}{Divergence relations are exactly maintained}
  \small%
  We can show that the discrete Maxwell equations on a Yee-cell
  maintain the divergence relations exactly:
  \begin{align*}
    \nabla_F\cdot\mvec{B}^{n+1/2} &= 0 \\
    \nabla_E\cdot\mvec{E}^n &= 0.
  \end{align*}
  There is an additional constraint of Maxwell equations in a plasma,
  that is, the current conservation:
  \begin{align*}
    \pfrac{\varrho_c}{t} + \nabla\cdot\mvec{J} = 0.
  \end{align*}
  where $\varrho_c$ is the charge density and $\mvec{J}$ is the
  current density. On the Yell-cell this becomes
  \begin{align*}
    \frac{\varrho_c^{n+1}-\varrho_v^{n}}{\Delta t}
    +
    \nabla_E\cdot\mvec{J}^{n+1/2} = 0.    
  \end{align*}
  One must ensure that current from particles is computed carefully to
  ensure that this expression is satified. See Esirkepov,
  Comp. Phys. Communications, {\bf 135} 144-153 (2001).
\end{frame}

%----------------------------------------------------------------
\begin{frame}{Methods for solution of hyperbolic/mixed PDEs}

  Hyperbolic (or mixed) PDEs appear everywhere in physics and
  engineering: Euler equations, Navier-Stokes equations, MHD
  equations, Einstein's equation of general relativity, shallow-water
  equations, ....%
  \mypause%
  \begin{itemize}
  \item Describe phenomena that travel with \emph{finite-speed}:
    clearly all fundamental physics must obey causality, hence
    hyperbolic systems are fundamental (though approximations of
    hyperbolic equations may violate finite-speed constraint).
    \mypause%
  \item Display very rich structure shocks and other discontinuities,
    instabilities, turbulence ... and in most problems, a mix of all!
    \mypause%
  \item Specialized methods are needed to solve such systems: naive
    algorithms can {\bf cause disaster}!
  \end{itemize}
  \mypause%
  My goal is to develop \emph{conceptual} understanding about such
  equations and numerical methods to solve them. Literature is too
  vast to cover in few lectures!

\end{frame}

\begin{frame}{Dissipation, dispersion and robustness}
  Typically, numerical methods trade between accuracy and
  robustness. Very accurate schemes are not too robust and highly
  robust schemes are typically not very accurate. Must find a balance.
  \mypause%
  \begin{itemize}
  \item To capture shocks and sharp gradient features some localized
    diffusion (dissipation) is needed via \emph{limiters} or some
    other means. Some dissipation may also be needed for stability.
  \item Dissipation typical leads to a loss of other properties: in
    particular, for Maxwell equations it can cause EM energy to
    \emph{decay}. For Euler equations high-k modes may get over-damped
    due to limiters: not good for turbulence problems.%
    \mypause%
  \item An ideal situation is to apply dissipation only where it is
    needed and use low-dissipation schemes elsewhere. However, this is
    easier said than done. How to determine where to apply dissipation
    is very tricky, specially in nonlinear complex flows. Ease to
    confuse physical features for numerical artifacts (and
    vice-versa!)
  \end{itemize}
\end{frame}

\begin{frame}{Hyperbolic PDEs describe phenomena that travel at finite
    speed}
  \small%
  An intuitive ``definition'' that we will initially work with before
  stating the mathematically rigorous definition:
  \begin{definition}[Hyperbolic PDEs ``Intuitive Definition'']
    A hyperbolic PDE is one in which all phenomena travel at a finite
    speed.
  \end{definition}

  \mypause%
  \vskip0.1in%
  Some prototypical examples we will look at more closely:
  \begin{itemize}
  \item Advection equation: simplest trivial \emph{linear} hyperbolic
    equation. Trivial but very important!
  \item Maxwell equation of electromagnetism: linear hyperbolic system
  \item Euler equations: probably historically the most important
    \emph{nonlinear hyperbolic system}. Basis of vast literature on
    numerical methods and basis for more complex equations: ideal MHD,
    (general) relativistic hydro/MHD, Navier-Stokes solvers, etc. Need
    to understand even if you want to follow literature and apply
    methods to your own problem.
  \end{itemize}
  
\end{frame}

\begin{frame}{Hyperbolic PDEs: rigorous definition, no reliance on
    linearization}
  Consider a system of conservation laws written as
  \begin{align*}
    \pfrac{\mvec{Q}}{t} + \pfrac{\mvec{F}}{x} = 0.
  \end{align*}
  where $\mvec{Q}$ is a vector of conserved quantities and
  $\mvec{F}(\mvec{Q})$ is a vector of fluxes. This system is called
  \emph{hyperbolic} if the flux Jacobian
  \begin{align*}
    \mvec{A} \equiv \pfrac{\mvec{F}}{\mvec{Q}}
  \end{align*}
  has \emph{real eigenvalues} and a \emph{complete set of linearly
    independent} eigenvectors. In multiple dimensions if $\mvec{F}_i$
  are fluxes in direction $i$ then we need to show that arbitrary
  linear combinations
  $\sum_i n_i {\partial\mvec{F}_i}/{\partial\mvec{Q}}$ have real
  eigenvalues and linearly independent set of eigenvectors.
  
\end{frame}

\begin{frame}{To compute eigensystem often easier to work in
    \emph{quasilinear form}}
  To derive eigensystem it is sometimes easier to work in
  non-conservative (quasi-linear) form of equations. Start with
  \begin{align*}
    \pfrac{\mvec{Q}}{t} + \pfrac{\mvec{F}}{x} = 0.
  \end{align*}
  and introduce an invertible transform $\mvec{Q} = \varphi(\mvec{V})$
  where $\mvec{V}$ are some other variables (for example: density,
  velocity and pressure). Then the system converts to
  \begin{align*}
    \pfrac{\mvec{V}}{t} +
    \underbrace{(\varphi^{\prime})^{-1} \mvec{A}\varphi^{\prime}}_{\mvec{B}}
    \pfrac{\mvec{V}}{x} = 0.
  \end{align*}
  Can easily show eigenvalues of $\mvec{A}$ are same as that of
  $\mvec{B}$ and right eigenvectors can be computed from
  $\varphi^{\prime} \mvec{r}_p$ and left eigenvectors from
  $\mvec{l}_p (\varphi^{\prime})^{-1}$.
\end{frame}

\begin{frame}{Burgers' equation}
  \footnotesize%
  Simplest nonlinear scalar equation, has a quadratic nonlinearity:
  \begin{align*}
    \pfrac{u}{t} + \frac{\partial}{\partial x}\bigg( \frac{1}{2} u^2
    \bigg) = 0.
  \end{align*}
  Eigenvalue $\lambda = u$. Note that locally, eigenvalue depends on
  the solution itself: leads to a situation in which after some finite
  time to the formation of a \emph{shock} as the ``characteristics''
  will intersect.
  \begin{figure}
    \setkeys{Gin}{width=0.5\linewidth,keepaspectratio}
    \incfig{burgers-char.pdf}
  \end{figure}  
\end{frame}

\begin{frame}{Burgers' equation: shock formation}

  \begin{columns}
    \begin{column}{0.5\textwidth}
      Due to varying characteristics speed (eigenvalue) the solution
      can ``pile up'' leading to the formation of a shock.
    \end{column}
    \begin{column}{0.5\textwidth}
      \begin{figure}
        \setkeys{Gin}{width=1.0\linewidth,keepaspectratio}
        \incfig{burgers-shock.png}        
      \end{figure}
    \end{column}
  \end{columns}

\end{frame}

\begin{frame}{Weak-solutions and entropy conditions}
  \footnotesize%
  At a shock the solution has a discontinuity. Hence, derivatives are
  not defined! Differential form of the equations break-down. We must
  use concept of weak-solutions in this case.%
  \vskip0.1in%
  \mypause%
  Let $\phi(x,t)$ is a compactly supported (i.e. zero outside some
  bounded region) smooth function (enough continuous
  derivatives). Then multiply conservation law
  \begin{align*}
    \int_0^\infty  \int_{-\infty}^\infty \phi(x,t)
    \bigg[\pfrac{\mvec{Q}}{t} + \pfrac{\mvec{F}}{x}\bigg]\thinspace
    dx\thinspace dt = 0
  \end{align*}
  by $\phi(x,t)$ and integrating by parts to get the \emph{weak-form}
  \begin{align*}
    \int_0^\infty  \int_{-\infty}^\infty 
    \bigg[\pfrac{\phi}{t} \mvec{Q} + \pfrac{\phi}{x} \mvec{F}\bigg]\thinspace
    dx\thinspace dt
    =
    -
    \int_{-\infty}^\infty \phi(x,0) \mvec{Q}(x,0) dx\thinspace dt.
  \end{align*}  
  \begin{definition}[Weak-solution]
    A function $\mvec{Q}(x,t)$ is said to be a weak-solution if it
    satisfies the weak-form for all compact, smooth $\phi(x,t)$.
  \end{definition}
  
\end{frame}

\begin{frame}{Weak-solutions and entropy conditions}
  Unfortunately, weak-solutions are not unique! Why does this happen?
  \vskip0.1in%
  \mypause In physical problems there is always some non-ideal effects
  (viscosity, Landau damping etc) that does not allow a genuine
  discontinuity to form. However, this ``viscous shock layer'' can be
  extremely thin compared to system size. Also, we know entropy must
  increase in the physical universe.

  \mypause%
  \vskip0.1in%
  This indicates we can recover uniqueness in two ways
  \begin{itemize}
  \item {\color{gray}{Add a viscous (diffusion) term and take limit of
      viscosity going to zero. (Generally not convenient for numerical
      work)}}%
  \mypause%
  \item Impose \emph{entropy condition}: construct an \emph{entropy}
    function such that it remains conserved for smooth solutions but
    \emph{increases} across a shock. Entropy is naturally suggested in
    most physical problems.
  \end{itemize}  
\end{frame}

\begin{frame}{Weak-solutions and entropy conditions}
  \small%
  When characteristics \emph{diverge} (right plot below) the
  weak-solution is not unique. A false ``shock'' solution also is a
  weak-solution. Imposing \emph{entropy condition} gives a
  \emph{rarefaction} wave seen in the right plot.
  \begin{columns}
    \begin{column}{0.5\textwidth}
      \begin{figure}
        \setkeys{Gin}{width=1.0\linewidth,keepaspectratio}
        \incfig{burgers-step-a.png}
      \end{figure}      
    \end{column}
    \begin{column}{0.5\textwidth}
      \begin{figure}
        \setkeys{Gin}{width=1.0\linewidth,keepaspectratio}
        \incfig{burgers-step-b.png}
      \end{figure}
    \end{column}
  \end{columns}  
\end{frame}

\begin{frame}{Euler equations of invicid fluids}
  \begin{align*}
    \frac{\partial}{\partial{t}}    
    \left[
    \begin{matrix}
      \rho \\
      \rho u \\
      \rho v \\
      \rho w \\
      \mathcal{E}
    \end{matrix}
    \right]
    +
    \frac{\partial}{\partial{x}}
    \left[
    \begin{matrix}
      \rho u \\
      \rho u^2 + p \\
      \rho uv \\
      \rho uw \\
      (\mathcal{E}+p)u
    \end{matrix}
    \right]
    =
    0    
  \end{align*}
  Here $\mathcal{E} = p/(\gamma-1) + \rho u^2/2$ is the total
  energy. Eigenvalues of this system are $\{u-c_s,u,u,u,u+c_s \}$
  where $c_s = \sqrt{\gamma p/rho}$ is the sound speed. See class
  notes for left/right eigenvectors.%
  \vskip0.1in%
  Note: in the limit $p\rightarrow 0$ all eigenvalues become $u$ and
  for cold-fluid ($p=0$) the system does not possess complete set of
  eigenvectors. (Cold fluid model is important to model dust, for
  example, in astrophysical systems or in say volcanic explosions).
\end{frame}

\begin{frame}{Euler equations: transport of kinetic energy}
  \footnotesize%
  The energy conservation equation for Euler equation is
  \begin{align*}
    \pfrac{\mathcal{E}}{t} + \nabla\cdot\left[(\mathcal{E}+p)\mvec{u} \right] = 0
  \end{align*}
  where
  \begin{align*}
    \mathcal{E} =
    \underbrace{\frac{p}{\gamma-1}}_{\textrm{IE}} +
    \underbrace{\frac{1}{2}\rho u^2}_{\textrm{KE}}
    .
  \end{align*}
  \mypause%
  We can derive instead a \emph{balance law} (not conservation law)
  for transport of KE
  \begin{align*}
    \frac{\partial}{\partial t}\left(\frac{1}{2}\rho u^2 \right)
    +
    \nabla\cdot\left(\frac{1}{2}\rho u^2 \mvec{u} \right)
    =
    -\mvec{u}\cdot\nabla p
    \color{gray}{ + \frac{q}{m}\rho \mvec{u}\cdot\mvec{\mvec{E}}}
  \end{align*}
  \mypause%
  For turbulence calculations it is important to ensure that in the
  \emph{numerics} exchange of kinetic and internal energy (and in case
  of plasma field-particle energy) is only via the RHS terms (pressure
  work and work done by electric field).%
  \vskip0.1in%
  Many shock-capturing and higher-order methods can mess this up for
  high-$k$ (short wavelength) modes due leading to incorrect energy
  spectra. (No Free Lunch Principle).
\end{frame}

\begin{frame}{Euler equations: shocks, rarefactions and contacts}
  In addition to shocks and rarefactions which we saw in Burgers's
  equation, Euler equations also support \emph{contact
    discontinuities}, across which density has a jump but not pressure
  or velocity.
  \begin{figure}
    \setkeys{Gin}{width=0.3\linewidth,keepaspectratio}
    \incfig{sod-density.png}
    \incfig{sod-pressure.png}
    \incfig{sod-vel.png}    
  \end{figure}    
\end{frame}

\begin{frame}{Ideal MHD equations}
  % \footnotesize%
  \small%
  Ideal MHD equations are very important to both fusion and
  astrophysical problems. Written in non-conservative form they are
  \begin{align*}
    \frac{\partial \rho}{\partial t}+ \mvec{u}\cdot\nabla\rho + \rho\nabla\cdot\mathbf{u}&=0 \\
    \frac{\partial \mathbf{u}}{\partial t}+\mathbf{u} \cdot \nabla
    \mathbf{u}
    +\frac{\nabla p}{\rho} &=
                             \color{blue}{\frac{1}{\mu_{0}\rho}(\nabla \times \mathbf{B}) \times \mathbf{B}} \\
    \frac{\partial p}{\partial t}+\mathbf{u} \cdot \nabla p+\gamma p
    \nabla \cdot \mathbf{u} &= 0 \\
    \color{blue}{\frac{\partial \mathbf{B}}{\partial t}-\nabla \times(\mathbf{u} \times \mathbf{B})} &= \color{blue}{0}
  \end{align*}
  with the constraint $\nabla\cdot\mvec{B} = 0$. The eigensystem is
  complicated to compute! (Try doing it yourself). Eigenvalues are
  $u\pm c_f$, $u\pm c_s$, $u\pm c_a$, and $u$ (7 eigenvalues for 8
  equations). Here $c_f$, $c_s$ are the fast/slow magnetosonic speeds
  and $c_a$ is the Alfven speed. See Ryu and Jones ApJ {\bf 442}
  228-258, 1995 (linked on website).
\end{frame}

\end{document}


\begin{frame}{}
\end{frame}

\begin{columns}
  
  \begin{column}{0.6\linewidth}
  \end{column}
  
  \begin{column}{0.4\linewidth}
    \includegraphics[width=\linewidth]{fig/Kinsey_2011_Pfus_vs_T.pdf}
  \end{column}
\end{columns}

% ----------------------------------------------------------------
