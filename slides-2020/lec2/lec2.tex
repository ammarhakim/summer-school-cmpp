\documentclass[pdf]{beamer}
\usepackage{hyperref}
%% AMS packages and font files
\usepackage{amsmath}
\usepackage{amsfonts}
\usepackage{amsthm}
\usepackage{stmaryrd}
\usepackage{caption}

\captionsetup{font=scriptsize,labelfont=scriptsize}
\footnotesize\small

\DeclareMathAlphabet{\mathpzc}{OT1}{pzc}{m}{it}

\newcommand{\eqr}[1]{Eq.\thinspace(#1)}
\newcommand{\pfrac}[2]{\frac{\partial #1}{\partial #2}}
\newcommand{\pfracc}[2]{\frac{\partial^2 #1}{\partial #2^2}}
\newcommand{\pfraca}[1]{\frac{\partial}{\partial #1}}
\newcommand{\pfracb}[2]{\partial #1/\partial #2}
\newcommand{\pfracbb}[2]{\partial^2 #1/\partial #2^2}
\newcommand{\spfrac}[2]{{\partial_{#1}} {#2}}
\newcommand{\mvec}[1]{\mathbf{#1}}
\newcommand{\gvec}[1]{\boldsymbol{#1}}
\newcommand{\script}[1]{\mathpzc{#1}}
\newcommand{\eep}{\mvec{e}_\phi}
\newcommand{\eer}{\mvec{e}_r}
\newcommand{\eez}{\mvec{e}_z}
\newcommand{\iprod}[2]{\langle{#1}\rangle_{#2}}
%\newcommand{\mypause}{\pause}
\newcommand{\mypause}{}
\newcommand{\gcs}{\nabla_{\mvec{x}}}
\newcommand{\gvs}{\nabla_{\mvec{v}}}
\newcommand{\gps}{\nabla_{\mvec{z}}}
\newcommand{\dtv}{\thinspace d^3\mvec{v}}
\newcommand{\dtx}{\thinspace d^3\mvec{x}}

\newtheorem{thm}{Theorem}
\newtheorem{lem}{Lemma}
\newtheorem{remark}{Remark}

\newtheorem{proposition}{Proposition}

\theoremstyle{definition}
\newtheorem{dfn}{Definition}
%\newtheorem{thm}{Theorem}
\newtheorem{proposed}{Proposed Answer}
\DeclareMathOperator{\spn}{span}

%% autoscaled figures
\newcommand{\incfig}{\centering\includegraphics}
\setkeys{Gin}{width=0.7\linewidth,keepaspectratio}

\usepackage{beamerthemesplit}
\setbeamercovered{transparent}
%\usetheme{lined}
%\usetheme{CambridgeUS}
%\usecolortheme{dolphin}
\usetheme{Madrid}
\usecolortheme{beaver}

\usefonttheme[onlymath]{serif}
\setbeamertemplate{navigation symbols}{}

% Setup TikZ
\usepackage{tikz}
\usetikzlibrary{arrows}
\tikzstyle{block}=[draw opacity=0.7,line width=1.4cm]

\title[{\tt }] {Computational Methods in Plasma Physics. Lecture II}%
\author[http://cmpp.rtfd.io]%
{Ammar H. Hakim\inst{1}}%

\institute[PPPL]
{ \inst{1} Princeton Plasma Physics Laboratory, Princeton, NJ %
}

\date[8/11/2020]{PPPL Graduate Summer School, 2020}

\begin{document}

\begin{frame}
  \titlepage
\end{frame}

\begin{frame}{Single particle motion in an electromagnetic field}
  \small%
  \begin{itemize}
  \item In the \emph{Particle-in-cell} (PIC) method the Vlasov-Maxwell
    equation is solved in the \emph{Lagrangian frame} in which the
    phase-space is represented by \emph{finite-sized}
    ``macro-particles''.
  \item In the Lagrangian frame the distribution
    function remains constants along \emph{characteristics} in
    phase-space.
  \item These characteristics satisfy the ODE of particles moving
    under Lorentz force law
    \begin{align*}
      \frac{d\mvec{x}}{dt} &= \mvec{v} \\
      \frac{d\mvec{v}}{dt} &= \frac{q}{m}(\mvec{E}(\mvec{x},t) + \mvec{v}\times\mvec{B}(\mvec{x},t))
    \end{align*}
  \item We will first focus on solving the equations-of-motion for
    the macro-particles, leaving solution of Maxwell equations and
    coupling to particles for Lecture 2.
  \end{itemize}
\end{frame}

\begin{frame}{Simple harmonic oscillator}
  \small%
  Consider first the simple harmonic oscillator
  \begin{align*}
    \frac{d^2z}{dt^2} = -\omega^2 z
  \end{align*}
  This has exact solution $z = a\cos(\omega t) + b\sin(\omega t)$,
  where $a$ and $b$ are arbitrary constants. How to solve this
  numerically? Write as a system of first-order ODEs
  \begin{align*}
    \frac{dz}{dt} = v; \quad \frac{dv}{dt} = -\omega^2 z
  \end{align*}
  Note that the coordinates $(z,v)$ label the \emph{phase-space} of
  the harmonic oscillator. Multiply the second equation by $v$ and use
  the first equation to get
  \begin{align*}
    \frac{d}{dt}\left(\frac{1}{2} v^2 + \frac{1}{2}\omega^2 z^2\right) = 0.
  \end{align*}
  This is the \emph{energy} and is \emph{conserved}.%
  
  \bf{Question: how to solve the ODE such that the energy is conserved
    by the \emph{discrete scheme}?}
\end{frame}

\begin{frame}{Harmonic oscillator: Forward Euler Scheme}
  First attempt: use the simplest possible scheme, replace derivatives
  with difference approximations
  \begin{align*}
    \frac{z^{n+1}-z^n}{\Delta t} = v^n; \quad \frac{v^{n+1}-v^n}{\Delta t} = -\omega^2 z^n
  \end{align*}
  or
  \begin{align*}
    z^{n+1} = z^n + \Delta t v^n; \quad v^{n+1} = v^n - \Delta t \omega^2 z^n
  \end{align*}
  This is the \emph{forward Euler} scheme. Lets check if the discrete
  scheme conserves energy:
  \begin{align*}
    (v^{n+1})^2 + \omega^2 (z^{n+1})^2 =
    (1+\omega^2 \Delta t^2)((v^{n})^2 + \omega^2 (z^{n})^2)
  \end{align*}
  The presence of the $\omega^2 \Delta t^2$ in the bracket spoils the
  conservation. So the forward Euler scheme \emph{does not} conserve
  energy. Also, note that the energy, in fact, is \emph{increasing}!
\end{frame}

\begin{frame}{Harmonic oscillator: Forward Euler Scheme}
  Closer look: write as a matrix equation
  \begin{align*}
    \left[
    \begin{matrix}
      z^{n+1} \\
      v^{n+1}
    \end{matrix}
    \right]
    =
    \underbrace{
    \left[
    \begin{matrix}
      1 & \Delta t \\
      -\omega^2 \Delta t & 1
    \end{matrix}
    \right]
    }_{\textrm{Jacobian},\ J}
    \left[
    \begin{matrix}
      z^{n} \\
      v^{n}
    \end{matrix}
    \right].                        
  \end{align*}
  Observe that the determinant of the Jacobian is
  $\det(J)=(1+\omega^2\Delta t^2)$ which is the same factor as appears
  in the energy relation. One may reasonably conjecture that when this
  determinant is one, then perhaps energy is conserved.

  \begin{block}{Volume Preserving Scheme}
    We will call say a scheme preserves \emph{phase-space} volume
    if the determinant of the Jacobian is $\det(J)=1$.
  \end{block}

\end{frame}

\begin{frame}{Harmonic oscillator: Mid-point Scheme}
  \small%
  Perhaps a better approximation will be obtained if we use
  \emph{averaged} values of $z,v$ on the RHS of the discrete
  equation:
  \begin{align*}
    \frac{z^{n+1}-z^n}{\Delta t} &= \frac{v^n +v^{n+1}}{2} \\
    \frac{v^{n+1}-v^n}{\Delta t} &= -\omega^2\frac{z^n + z^{n+1}}{2}
  \end{align*}
  This is an \emph{implicit} method as the solution at the next
  time-step depends on the old as well as the next time-step
  values. In this simple case we can explicitly write the update in a
  matrix form as
  \begin{align*}
    \left[
    \begin{matrix}
      z^{n+1} \\
      v^{n+1}
    \end{matrix}
      \right]
      =
      \frac{1}{1+\omega^2\Delta t^2/4}
    \left[
      \begin{matrix}
        1-\omega^2\Delta t^2/4  & \Delta t \\
        -\omega^2 \Delta t & 1-\omega^2\Delta t^2/4
      \end{matrix}
    \right]
    \left[
    \begin{matrix}
      z^{n} \\
      v^{n}
    \end{matrix}
    \right].                        
  \end{align*}
  For this scheme $\det(J)=1$. So the mid-point scheme conserves
  phase-space volume! Some algebra also shows that
  \begin{align*}
    (v^{n+1})^2 + \omega^2 (z^{n+1})^2 = (v^{n})^2 + \omega^2 (z^{n})^2
  \end{align*}
  showing that energy is also conserved by the mid-point scheme.
\end{frame}

\begin{frame}{Harmonic oscillator: Mid-point Scheme is symplectic}
  A more stringent constraint on a scheme for the simple harmonic
  oscillator is that it be \emph{symplectic}. To check if a scheme is
  symplectic one checks to see if
  \begin{align*}
    J^T\sigma J = \sigma
  \end{align*}
  where $\sigma$ is the \emph{unit symplectic matrix}
  \begin{align*}
    \sigma
    =
    \left[
      \begin{matrix}
        0 & 1 \\
        -1 & 0
      \end{matrix}
    \right]    
  \end{align*}
  Turns out that the mid-point scheme for the harmonic oscillator is
  also symplectic. Note that if a scheme conserves phase-space
  volume, it \emph{need not} be symplectic.
\end{frame}

\begin{frame}{Accuracy and Stability}
  \small%
  To study the stability, accuracy and convergence of a scheme one usually
  looks at the first order ODE
  \begin{align*}
    \frac{dz}{dt} = -\gamma z
  \end{align*}
  where $\gamma = \lambda + i\omega$ is the complex frequency. The
  exact solution to this equation is $z(t) = z_0 e^{-\gamma t}$. The
  solution has damped/growing modes ($\lambda>0$ or $\lambda<0$) as
  well as oscillating modes.

  \begin{itemize}
    \item The forward Euler scheme for this equation is
      \begin{align*}
        z^{n+1} = z^n - \Delta t \gamma z^n = (1-\Delta t\gamma) z^n.
      \end{align*}
    \item The mid-point scheme for this equation is
      \begin{align*}
        z^{n+1} = \left(\frac{1-\gamma\Delta t/2}{1+\gamma\Delta t/2}\right) z^n
      \end{align*}
    \end{itemize}
\end{frame}

\begin{frame}{Accuracy and Stability}
  We can determine how \emph{accurate} the scheme is by looking at how
  many terms the scheme matches the Taylor series expansion of the
  exact solution:
  \begin{align*}
    z(t^{n+1}) = z(t^n)
    \left(
    1 - \gamma\Delta t + \frac{1}{2}\gamma^2 \Delta t^2 - \frac{1}{6}\gamma^3 \Delta t^3 + \ldots
    \right)
  \end{align*}

  \begin{itemize}
  \item The forward Euler scheme matches the \emph{first two terms}
    \begin{align*}
      z^{n+1} = z^n (1-\Delta t\gamma)
    \end{align*}
  \item The mid-point scheme matches the \emph{first three terms}
    \begin{align*}
      z^{n+1} = z^n \left(
      1-\Delta t\gamma - \frac{1}{2}\gamma^2 \Delta t^2  - \frac{1}{4} \gamma^3 \Delta t^3 + \ldots
      \right)
    \end{align*}
  \end{itemize}
\end{frame}

\begin{frame}{Accuracy and Stability}
  We can determine if the scheme is \emph{stable} by looking at the
  amplification factor $| z^{n+1}/z^n|$. Note that for damped modes
  ($\lambda>0$) this quantity \emph{decays} in time, while for purely
  oscillating modes ($\lambda = 0$) this quantity remains
  \emph{constant}.

  
  \begin{itemize}
  \item The amplification factor for the forward Euler scheme in the
    absence of damping is $1+\omega^2\Delta t^2 > 1$, hence this
    scheme is \emph{unconditionally unstable}.
  \item The amplification factor for the mid-point scheme in the
    absence of damping is exactly 1, showing that the mid-point scheme
    is \emph{unconditionally stable}, that is, one can take as large
    time-step one wants without the scheme ``blowing up''. Of course,
    the errors will increase with larger $\Delta t$.
  \end{itemize}
\end{frame}

\begin{frame}{Runge-Kutta schemes}
  \begin{itemize}
  \item Even though the forward Euler scheme is unconditionally
    unstable, we can use it to construct other schemes that \emph{are}
    stable and are also more accurate (than first order).
  \item For example, a class of Runge-Kutta schemes can be written as
    a combination of forward Euler updates. In particular, the
    \emph{strong stability preserving} schemes are important when
    solving hyperbolic equations. Note that these RK schemes will
    \emph{not} conserve energy for the harmonic oscillator, but
    \emph{decay} it.
  \item Other multi-stage Runge-Kutta schemes can be constructed that
    allow very large time-steps for diffusive processes, for example,
    that come about when time-stepping diffusion equations.
  \end{itemize}
\end{frame}

\begin{frame}{Simple harmonic oscillator}
  We looked at
  \begin{align*}
    \frac{d^2z}{dt^2} = -\omega^2 z
  \end{align*}
  and wrote it as system of first-order ODEs
  \begin{align*}
    \frac{dz}{dt} = v; \quad \frac{dv}{dt} = -\omega^2 z
  \end{align*}
  Now introduce energy-angle coordinates
  \begin{align*}
    \omega z = E\sin\theta; \quad v = E\cos\theta
  \end{align*}
  then $E^2 = \omega^2 z^2 + v^2 \equiv E_0^2$ is a constant as we
  showed before. Using these expressions we get the very simple ODE
  $\dot{\theta} = \omega$. This shows that in phase-space
  $(v,\omega z)$ the motion is with uniform angular speed along a
  circle.
\end{frame}

\begin{frame}{Simple harmonic oscillator: Phase-errors}
  \small%
  The mid-point scheme had
  \begin{align*}
    (v^{n+1})^2 + \omega^2 (z^{n+1})^2 = (v^{n})^2 + \omega^2 (z^{n})^2
    = E_0^2
  \end{align*}
  which means that the mid-point scheme gets the energy coordinate
  \emph{exactly} correct. However, we have
  \begin{align*}
    \tan\theta^{n+1} = \frac{\omega z^{n+1}}{v^{n+1}}.
  \end{align*}
  Using the expressions for the scheme and Taylor expanding in
  $\Delta t$ we get
  \begin{align*}
    \tan\theta^{n+1}
    =
    \tan\theta^n +
    \frac{\omega E_0^2}{(v^{n})^2}\Delta t +
    \frac{\omega^3 z^n E_0^2}{(v^n)^3}\Delta t^2
    + O(\Delta t^3)
  \end{align*}
  The first three terms match the Taylor expansion of the exact
  solution $\tan(\theta^n+\omega\Delta t)$ and the last term is the
  \emph{phase-error}.
\end{frame}

\begin{frame}{Single particle motion in an electromagnetic field}
  \small%
  \begin{itemize}
  \item In PIC method the Vlasov-Maxwell equation is solved in the
    \emph{Lagrangian frame}: the phase-space is represented by
    \emph{finite-sized} ``macro-particles''.
  \item In the Lagrangian frame the distribution
    function remains constants along \emph{characteristics} in
    phase-space.
  \item These characteristics satisfy the ODE of particles moving
    under Lorentz force law
    \begin{align*}
      \frac{d\mvec{x}}{dt} &= \mvec{v} \\
      \frac{d\mvec{v}}{dt} &= \frac{q}{m}(\mvec{E}(\mvec{x},t) + \mvec{v}\times\mvec{B}(\mvec{x},t))
    \end{align*}
  \item In the absence of an electric field, the kinetic energy must
    be conserved
    \begin{align*}
      \frac{1}{2} |\mvec{v}|^2 = \textrm{constant}.
    \end{align*}
    This is independent of the spatial or time dependence of the
    magnetic field. Geometrically this means that in the absence of an
    electric field the velocity vector rotates and its tip always lies
    on a sphere.
  \end{itemize}
\end{frame}

\begin{frame}{Single particle motion in an electromagnetic field}
  \small%
  \begin{itemize}
  \item A mid-point scheme for this equation system would look like
    \begin{align*}
    \frac{\mvec{x}^{n+1}-\mvec{x}^n}{\Delta t} &= \frac{\mvec{v}^{n+1}+\mvec{v}^n}{2} \\
    \frac{\mvec{v}^{n+1}-\mvec{v}^n}{\Delta t} &= \frac{q}{m}
      \big(
        \overline{\mvec{E}}(\mvec{x},t) +
        \frac{\mvec{v}^{n+1}+\mvec{v}^n}{2}\times\overline{\mvec{B}}(\mvec{x},t)
      \big)
    \end{align*}
    The overbars indicate some averaged electric and magnetic fields
    evaluated from the new and old positions. In general, this would
    make the scheme nonlinear!

  \item Instead, we will use a \emph{staggered} scheme in which the
    position and velocity are staggered by half a time-step.
    \begin{align*}
    \frac{\mvec{x}^{n+1}-\mvec{x}^n}{\Delta t} &= \mvec{v}^{n+1/2} \\
    \frac{\mvec{v}^{n+1/2}-\mvec{v}^{n-1/2}}{\Delta t} &= \frac{q}{m}
      \big(
        \mvec{E}(\mvec{x}^n,t^n) +
        \frac{\mvec{v}^{n+1/2}+\mvec{v}^{n-1/2}}{2}\times\mvec{B}(\mvec{x}^n,t^n)
      \big)
    \end{align*}
  \end{itemize}
\end{frame}

\begin{frame}{The Boris algorithm for the staggered scheme}
  \footnotesize%
  The velocity update formula is
  \begin{align*}
    \frac{\mvec{v}^{n+1/2}-\mvec{v}^{n-1/2}}{\Delta t} &= \frac{q}{m}
      \big(
        \mvec{E}(\mvec{x}^n,t^n) +
        \frac{\mvec{v}^{n+1/2}+\mvec{v}^{n-1/2}}{2}\times\mvec{B}(\mvec{x}^n,t^n)
      \big)
  \end{align*}
  This appears like an implicit method: most obvious is to construct a
  linear $3\times 3$ system of equations and invert them to determine
  $\mvec{v}^{n+1}$. Puzzle to test your vector-identity foo: find
  $\mvec{A}$ if $\mvec{A} = \mvec{R}+\mvec{A}\times\mvec{B}$.

  The Boris algorithm updates this equation in three steps:
  \begin{align*}
    \mvec{v}^- &= \mvec{v}^{n-1/2} + \frac{q}{m}\mvec{E}^n\frac{\Delta t}{2} \\
    \frac{\mvec{v}^+- \mvec{v}^-}{\Delta t}
    &=
      \frac{q}{2 m}(\mvec{v}^++\mvec{v}^-)\times\mvec{B}^n \\
    \mvec{v}^{n+1/2}
    &=
      \mvec{v}^{+} + \frac{q}{m}\mvec{E}^n\frac{\Delta t}{2} \\    
  \end{align*}
  Convince yourself that this is indeed equivalent to the staggered
  expression above. So we have two electric field updates with half
  time-steps and a rotation due to the magnetic field. Once the
  updated velocity is computed, we can trivially compute the updated
  positions.
\end{frame}

\begin{frame}{The Boris algorithm for the staggered scheme}
  How to do the rotation? The Boris algorithm does this in several
  steps:
  \begin{itemize}
  \item Compute the $\mvec{t}$ and $\mvec{s}$ vectors as follows
    \begin{align*}
      \mvec{t} &= \frac{q\mvec{B}}{m}\frac{\Delta t}{2} \\
      \mvec{s} &= \frac{2\mvec{t}}{1+|\mvec{t}|^2}
    \end{align*}
  \item Compute $\mvec{v}' = \mvec{v}^- + \mvec{v}^-\times\mvec{t}$
    and finally $\mvec{v}^+ = \mvec{v}^-+\mvec{v}'\times\mvec{s}$.
  \end{itemize}
  See Birdsall and Langdon text book Section 4-3 and 4-4 and figure
  4-4a. Easily extended to relativistic case.%
  \vskip0.1in%
  Note that in the absence of an electric field the Boris algorithm
  conserves kinetic energy.
\end{frame}

\begin{frame}{Why is the Boris algorithm so good? Can one do better?}
  \small%
  See paper by Qin at. al. Phys. Plasmas, {\bf 20}, 084503 (2013) in
  which it is shown that the Boris algorithm \emph{conserves
    phase-space volume}. However, they also show that the Boris
  algorithm is \emph{not} symplectic.
  \begin{itemize}
  \item The relativistic Boris algorithm does not properly compute the
    $\mvec{E}\times\mvec{B}$ velocity. This can be corrected. For
    example Vay, Phys. Plasmas, {\bf 15}, 056701 (2008). The Vay
    algorithm however, breaks the phase-space volume preserving
    property of the Boris algorithm.
  \item Higuera and Cary, Phys. Plasmas, {\bf 24}, 052104 (2017)
    showed how to compute the correct $\mvec{E}\times\mvec{B}$ drift
    velocity and restore volume preserving property. Seems this is
    probably the current-best algorithm for updating Lorentz
    equations.
  \item The saga for better particle push algorithms is not over! For
    example, an active area of research is to discover good algorithms
    for \emph{asymptotic} systems, for example, when gyroradius is
    much smaller than gradient length-scales or gyrofrequency is much
    higher than other time-scales in the system. Common in most
    magnetized plasmas.
  \end{itemize}  
\end{frame}

% ----------------------------------------------------------------

\end{document}


\begin{frame}{}
\end{frame}

% ----------------------------------------------------------------
