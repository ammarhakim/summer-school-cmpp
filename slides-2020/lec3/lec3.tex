\documentclass[pdf]{beamer}
\usepackage{hyperref}
%% AMS packages and font files
\usepackage{amsmath}
\usepackage{amsfonts}
\usepackage{amsthm}
\usepackage{stmaryrd}
\usepackage{caption}

\captionsetup{font=scriptsize,labelfont=scriptsize}
\footnotesize\small

\DeclareMathAlphabet{\mathpzc}{OT1}{pzc}{m}{it}

\newcommand{\eqr}[1]{Eq.\thinspace(#1)}
\newcommand{\pfrac}[2]{\frac{\partial #1}{\partial #2}}
\newcommand{\pfracc}[2]{\frac{\partial^2 #1}{\partial #2^2}}
\newcommand{\pfraca}[1]{\frac{\partial}{\partial #1}}
\newcommand{\pfracb}[2]{\partial #1/\partial #2}
\newcommand{\pfracbb}[2]{\partial^2 #1/\partial #2^2}
\newcommand{\spfrac}[2]{{\partial_{#1}} {#2}}
\newcommand{\mvec}[1]{\mathbf{#1}}
\newcommand{\gvec}[1]{\boldsymbol{#1}}
\newcommand{\script}[1]{\mathpzc{#1}}
\newcommand{\eep}{\mvec{e}_\phi}
\newcommand{\eer}{\mvec{e}_r}
\newcommand{\eez}{\mvec{e}_z}
\newcommand{\iprod}[2]{\langle{#1}\rangle_{#2}}
%\newcommand{\mypause}{\pause}
\newcommand{\mypause}{}
\newcommand{\gcs}{\nabla_{\mvec{x}}}
\newcommand{\gvs}{\nabla_{\mvec{v}}}
\newcommand{\gps}{\nabla_{\mvec{z}}}
\newcommand{\dtv}{\thinspace d^3\mvec{v}}
\newcommand{\dtx}{\thinspace d^3\mvec{x}}

\newtheorem{thm}{Theorem}
\newtheorem{lem}{Lemma}
\newtheorem{remark}{Remark}

\newtheorem{proposition}{Proposition}

\theoremstyle{definition}
\newtheorem{dfn}{Definition}
%\newtheorem{thm}{Theorem}
\newtheorem{proposed}{Proposed Answer}
\DeclareMathOperator{\spn}{span}

%% autoscaled figures
\newcommand{\incfig}{\centering\includegraphics}
\setkeys{Gin}{width=0.7\linewidth,keepaspectratio}

\usepackage{beamerthemesplit}
\setbeamercovered{transparent}
%\usetheme{lined}
%\usetheme{CambridgeUS}
%\usecolortheme{dolphin}
\usetheme{Madrid}
\usecolortheme{beaver}

\usefonttheme[onlymath]{serif}
\setbeamertemplate{navigation symbols}{}

% Setup TikZ
\usepackage{tikz}
\usetikzlibrary{arrows}
\tikzstyle{block}=[draw opacity=0.7,line width=1.4cm]

\title[{\tt }] {Computational Methods in Plasma Physics. Lecture III}%
\author[http://cmpp.rtfd.io]%
{Ammar H. Hakim\inst{1}}%

\institute[PPPL]
{ \inst{1} Princeton Plasma Physics Laboratory, Princeton, NJ %
}

\date[8/12/2020]{PPPL Graduate Summer School, 2020}

\begin{document}

\begin{frame}
  \titlepage
\end{frame}

\begin{frame}{Solving Maxwell equations}
  \small%
  Besides pushing particles in electromagnetic fields, we need to
  compute these fields self-consistently from currents and charges by
  solving Maxwell equations. First consider Maxwell equations in
  vacuum:
  \begin{align*}
    \frac{\partial \mvec{B}}{\partial t} + \nabla\times\mvec{E} &= 0 \\
    \epsilon_0\mu_0\frac{\partial \mvec{E}}{\partial t} -
    \nabla\times\mvec{B} &= 0
  \end{align*}
  For these we have the conservation laws
  \begin{align*}
    \frac{d}{dt} \int_\Omega \mvec{E}\times\mvec{B}  \dtx &= 0 \\
    \frac{d}{dt} \int_\Omega \left( \frac{\epsilon_0}{2} |\mvec{E}|^2 + \frac{1}{2\mu_0} |\mvec{B}|^2 \right) \dtx &= 0.
  \end{align*}
  Note these are \emph{global} conservation laws and one can instead
  also write \emph{local} conservation laws that include momentum and
  energy flux terms. How to solve these equations efficiently and
  maintain (some) conservation and geometric properties?
\end{frame}

\begin{frame}{Solving Maxwell equations}
  \begin{itemize}
  \item Maxwell equations have a very special geometric structure. The
    electric field $\mvec{E}$ is a \emph{vector} while the magnetic
    field $\mvec{B}$ is a \emph{bivector} (this is disguised in the
    usual formulations of Maxwell equations).
  \item (In spacetime formulations the complete electromagnetic field
    is represented as a single bivector in 4D space-time).
  \item The fact that we are dealing with two objects of
    \emph{different} geometric types indicates that the discrete
    Maxwell equations should also inherit this somehow.
  \item The Yee algorithm, often called the \emph{finite-difference
      time-domain} algorithm is the most successful (and simple)
    algorithm that accounts of this geometric structure. It is
    implemented in most PIC codes, though recent research has focused
    on structure preserving finite-element and other methods.
  \end{itemize}
\end{frame}

\begin{frame}{Solving Maxwell equations: The Yee-cell}
  \begin{figure}
    \incfig{Yee-cube.png}
    \caption{Standard Yee-cell. Electric field components (vectors)
      are located on edges while magnetic field components
      (bivectors) are located on faces.}
  \end{figure}  
\end{frame}

\begin{frame}{Solving Maxwell equations: The Yee-cell}
  \small%
  On the Yee-cell the difference approximation to Maxwell equations
  ``falls out'', almost like magic. The updates are staggered in time
  and use two \emph{different} discrete curl operators:
  \begin{align*}
    \mvec{B}^{n+1/2} &= \mvec{B}^{n-1/2} - \Delta t\thinspace \nabla_E\times\mvec{E}^{n} \\    
    \mvec{E}^{n+1} &= \mvec{E}^n + \Delta t/c^2\thinspace \nabla_F\times\mvec{B}^{n+1/2}
  \end{align*}
  Here the symbols $\nabla_F\times$ and $\nabla_E\times$ are the
  discrete curl operators:
  \begin{itemize}
  \item The first takes \emph{face-centered} magnetic field and
    computes it curl. This operator \emph{puts the result on cell
      edges}.
  \item The second takes \emph{edge-centered} electric field and
    computes it curl. This operator \emph{puts the result on cell
      faces}.
  \item The structure of Yell-cell also indicates that \emph{currents}
    must be co-located with the electric field and computed at half
    time-steps.
  \end{itemize}
  This duality neatly reflects the underlying geometry of Maxwell
  equations. The staggering in time reflects the fact that in 4D the
  electromagnetic field is a bivector in spacetime.
\end{frame}

\begin{frame}{Divergence relations are exactly maintained}
  \small%
  We can show that the discrete Maxwell equations on a Yee-cell
  maintain the divergence relations exactly:
  \begin{align*}
    \nabla_F\cdot\mvec{B}^{n+1/2} &= 0 \\
    \nabla_E\cdot\mvec{E}^n &= 0.
  \end{align*}
  There is an additional constraint of Maxwell equations in a plasma,
  that is, the current conservation:
  \begin{align*}
    \pfrac{\varrho_c}{t} + \nabla\cdot\mvec{J} = 0.
  \end{align*}
  where $\varrho_c$ is the charge density and $\mvec{J}$ is the
  current density. On the Yell-cell this becomes
  \begin{align*}
    \frac{\varrho_c^{n+1}-\varrho_v^{n}}{\Delta t}
    +
    \nabla_E\cdot\mvec{J}^{n+1/2} = 0.    
  \end{align*}
  One must ensure that current from particles is computed carefully to
  ensure that this expression is satified. See Esirkepov,
  Comp. Phys. Communications, {\bf 135} 144-153 (2001).
\end{frame}

\begin{frame}{Maxwell equations are hyperbolic: large class of such
    equations}

  A vast class of partial differential equations are
  \emph{hyperbolic}.
  \begin{itemize}
    \item Consider
      \begin{align*}
        \pfrac{\mvec{Q}}{t} + \pfrac{\mvec{F}}{x} = 0
      \end{align*}
      where $\mvec{Q}$ is a vector of \emph{conserved} quantities and
      $\mvec{F} = \mvec{F}(\mvec{Q})$ is the vector of
      \emph{fluxes}.
    \item Compute the \emph{flux-jacobian}
      \begin{align*}
        \mvec{A} = \pfrac{\mvec{F}}{\mvec{Q}}.
      \end{align*}
    \item Now compute the eigenvalues and eigenvectors of
      $\mvec{A}$. If these eigenvalues are \emph{real} and
      eigenvectors are \emph{complete} then this system is called
      \emph{hyperbolic}.
    \end{itemize}
  
\end{frame}

\begin{frame}{Example 1: Maxwell equations}
  \small
  Consider the 1D source-free Maxwell equations
  \begin{align*}
    \frac{\partial }{\partial t}
    \left[
    \begin{matrix}
      E_y \\
      B_z
    \end{matrix}
    \right]
    +
    \frac{\partial }{\partial x}
    \left[
    \begin{matrix}
      c^2B_z \\
      E_y
    \end{matrix}
    \right]
    =
    0.
  \end{align*}
  The flux-jacobian is
  \begin{align*}
    \mvec{A} =
    \left[
    \begin{matrix}
      0 & c^2 \\
      1 & 0
    \end{matrix}
    \right]. 
  \end{align*}
  A simple calculation shows that the eigenvalues are $-c,c$ and
  eigenvectors
  \begin{align*}
    \mvec{r}^1 =
    \left[
    \begin{matrix}
      1 \\
      -1/c
    \end{matrix}
    \right]
    \qquad
    \mvec{r}^2 =
    \left[
    \begin{matrix}
      1 \\
      1/c
    \end{matrix}
    \right].
  \end{align*}
\end{frame}

\begin{frame}{Example 2: Euler equations}
  \small Consider the ideal fluid equations (Euler equations)
  \begin{align*}
  \frac{\partial}{\partial{t}}
  \left[
    \begin{matrix}
      \rho \\
      \rho u \\
      E
    \end{matrix}
  \right]
  +
  \frac{\partial}{\partial{x}}
  \left[
    \begin{matrix}
      \rho u \\
      \rho u^2 + p \\
      (E+p)u
    \end{matrix}
  \right]
  =
  0
  \end{align*}
  Where, $E = \rho u^2/2 + p/(\gamma-1)$ is the fluid energy. This is
  also hyperbolic, with eigenvalues $u \pm c_s$ where
  $c_s = \sqrt{\gamma p/\rho}$ is the sound speed.%
  \vskip0.1in%
  Note that the Euler equations are a \emph{nonlinear} hyperbolic
  system. Other examples: \emph{ideal MHD} equations, relativistic
  fluid equations, Einstein field equation for gravitation, ...
\end{frame}

\begin{frame}{Hyperbolic problems need special methods to solve}
  Major feature of hyperbolic equations are the existence of shocks
  and other nonlinear features (rarefactions, compression waves,
  contact discontinuities).

  To handle these properly one needs to account for the structure of
  the equation as revealed by the eigenvalues and eigenvectors.
  \begin{itemize}
  \item These \emph{shock capturing} methods were originally developed
    in aerospace engineering literature to solve problems of transonic
    and supersonic flight and reentry vehicles.
  \item They are built around the concept of a \emph{Reimann solver}
    that approximate the nonlinear structure at each cell interface.
  \item In fusion physics, these are not yet widely used. However,
    they are ubiquitous in astrophysical plasma physics.
  \end{itemize}
  Typically, the general belief is that the the evolution of the
  plasma in a fusion machine is too slow to justify shock capturing
  methods. 
\end{frame}

\begin{frame}{Maxwell equations}
  \small
  Consider the 1D source-free Maxwell equations
  \begin{align*}
    \frac{\partial }{\partial t}
    \left[
    \begin{matrix}
      E_y \\
      B_z
    \end{matrix}
    \right]
    +
    \frac{\partial }{\partial x}
    \left[
    \begin{matrix}
      c^2B_z \\
      E_y
    \end{matrix}
    \right]
    =
    0.
  \end{align*}
  Basic idea is to transform the equation into uncoupled advection
  equations for the \emph{Riemann variables}. This is always possible
  in 1D for \emph{linear hyperbolic} systems. For the above system,
  multiply the second equation by $c$ and add and subtract from the
  first equation to get
  \begin{align*}
    \frac{\partial }{\partial t}\left(E_y + c B_z\right) + c \frac{\partial }{\partial x}\left(E_y + c B_z\right) &= 0 \\
    \frac{\partial }{\partial t}\left(E_y - c B_z\right) - c \frac{\partial }{\partial x}\left(E_y - c B_z\right) &= 0.
  \end{align*}
  Note that these are two uncoupled passive advection equations for
  the variables $w^\pm = E_y \pm c B_z$ with advection speeds $\pm
  c$.
\end{frame}

\begin{frame}{Finite-Volume method for Maxwell equations}
  \small%
  \begin{itemize}
  \item Instead of using the Yee-cell one can solve Maxwell equations
    using finite-volume methods developed in aerospace and fluid
    mechanics.
  \item In these schemes one uses the local characterstic direction to
    ``upwind'' values at cell faces, adding stability to the scheme
    when simulating small-scale features.
  \item Finite-volume schemes are as cheap (or expensive) as Yell-cell
    based FDTD schemes and are also easy to implement for Maxwell
    equations. However, they suffer from some disadvantages.
  \item First, it is very hard to ensure divergence relations are
    maintained. One needs to correct the divergence errors somehow, by
    adding some additional equations to the system.
  \item Energy is not conserved by finite-volume schemes that use
    upwinding. Special choices of basis-functions can be used to
    construct energy conserving schemes, but these have other issues
    (also shared by FDTD schemes).
  \item However, in some situations, FV based Maxwell solvers are
    useful and have been successfully applied in many large-scale
    problems.
  \end{itemize}
\end{frame}  

% ----------------------------------------------------------------  

\begin{frame}{Choice of numerical fluxes for Maxwell equations impacts
    energy conservation}

  The electromagnetic energy is given by
  \begin{align*}
    \mathcal{E} = \frac{\epsilon_0}{2} E_y^2 + \frac{1}{2\mu_0} B_z^2
  \end{align*}
  Notice that this is the $L_2$ norm of the electromagnetic
  field. 
  \begin{itemize}
  \item Hence, as we will show tomorrow, if we use upwinding to
    compute numerical fluxes, the \emph{electromagnetic energy will
      decay}.
  \item If we use central fluxes (average left/right values) then the
    EM energy will remain conserved by the \emph{time-continuous}
    scheme. However, the Runge-Kutta time-stepping will add small
    diffusion that will decay the total energy a little.
  \item However, the energy decay rate will be \emph{independent} of
    the spatial resolution and will reduce with smaller time-steps.
  \end{itemize}
\end{frame}

% ----------------------------------------------------------------

\end{document}


\begin{frame}{}
\end{frame}

% ----------------------------------------------------------------
