\documentclass[pdf]{beamer}
\usepackage{hyperref}
%% AMS packages and font files
\usepackage{amsmath}
\usepackage{amsfonts}
\usepackage{amsthm}
\usepackage{stmaryrd}
\usepackage{caption}

\captionsetup{font=scriptsize,labelfont=scriptsize}
\footnotesize\small

\DeclareMathAlphabet{\mathpzc}{OT1}{pzc}{m}{it}

\newcommand{\eqr}[1]{Eq.\thinspace(#1)}
\newcommand{\pfrac}[2]{\frac{\partial #1}{\partial #2}}
\newcommand{\pfracc}[2]{\frac{\partial^2 #1}{\partial #2^2}}
\newcommand{\pfraca}[1]{\frac{\partial}{\partial #1}}
\newcommand{\pfracb}[2]{\partial #1/\partial #2}
\newcommand{\pfracbb}[2]{\partial^2 #1/\partial #2^2}
\newcommand{\spfrac}[2]{{\partial_{#1}} {#2}}
\newcommand{\mvec}[1]{\mathbf{#1}}
\newcommand{\gvec}[1]{\boldsymbol{#1}}
\newcommand{\script}[1]{\mathpzc{#1}}
\newcommand{\eep}{\mvec{e}_\phi}
\newcommand{\eer}{\mvec{e}_r}
\newcommand{\eez}{\mvec{e}_z}
\newcommand{\iprod}[2]{\langle{#1}\rangle_{#2}}
%\newcommand{\mypause}{\pause}
\newcommand{\mypause}{}
\newcommand{\gcs}{\nabla_{\mvec{x}}}
\newcommand{\gvs}{\nabla_{\mvec{v}}}
\newcommand{\gps}{\nabla_{\mvec{z}}}
\newcommand{\dtv}{\thinspace d^3\mvec{v}}
\newcommand{\dtx}{\thinspace d^3\mvec{x}}

\newtheorem{thm}{Theorem}
\newtheorem{lem}{Lemma}
\newtheorem{remark}{Remark}

\newtheorem{proposition}{Proposition}

\theoremstyle{definition}
\newtheorem{dfn}{Definition}
%\newtheorem{thm}{Theorem}
\newtheorem{proposed}{Proposed Answer}
\DeclareMathOperator{\spn}{span}

%% autoscaled figures
\newcommand{\incfig}{\centering\includegraphics}
\setkeys{Gin}{width=0.7\linewidth,keepaspectratio}

\usepackage{beamerthemesplit}
\setbeamercovered{transparent}
%\usetheme{lined}
%\usetheme{CambridgeUS}
%\usecolortheme{dolphin}
\usetheme{Madrid}
\usecolortheme{beaver}

\usefonttheme[onlymath]{serif}
\setbeamertemplate{navigation symbols}{}

% Setup TikZ
\usepackage{tikz}
\usetikzlibrary{arrows}
\tikzstyle{block}=[draw opacity=0.7,line width=1.4cm]

\title[{\tt }] {Computational Methods in Plasma Physics. Lecture II}%
\author[http://cmpp.rtfd.io]%
{Ammar H. Hakim\inst{1}}%

\institute[PPPL]
{ \inst{1} Princeton Plasma Physics Laboratory, Princeton, NJ %
}

\date[8/13/2019]{PPPL Graduate Summer School, 2019}

\begin{document}

\begin{frame}
  \titlepage
\end{frame}

\begin{frame}{Simple harmonic oscillator}
  We looked at
  \begin{align*}
    \frac{d^2z}{dt^2} = -\omega^2 z
  \end{align*}
  and wrote it as system of first-order ODEs
  \begin{align*}
    \frac{dz}{dt} = v; \quad \frac{dv}{dt} = -\omega^2 z
  \end{align*}
  Now introduce energy-angle coordinates
  \begin{align*}
    \omega z = E\sin\theta; \quad v = E\cos\theta
  \end{align*}
  then $E^2 = \omega^2 z^2 + v^2 \equiv E_0^2$ is a constant as we
  showed before. Using these expressions we get the very simple ODE
  $\dot{\theta} = \omega$. This shows that in phase-space
  $(v,\omega z)$ the motion is with uniform angular speed along a
  circle.
\end{frame}

\begin{frame}{Simple harmonic oscillator: Phase-errors}
  \small%
  The mid-point scheme had
  \begin{align*}
    (v^{n+1})^2 + \omega^2 (z^{n+1})^2 = (v^{n})^2 + \omega^2 (z^{n})^2
    = E_0^2
  \end{align*}
  which means that the mid-point scheme gets the energy coordinate
  \emph{exactly} correct. However, we have
  \begin{align*}
    \tan\theta^{n+1} = \frac{\omega z^{n+1}}{v^{n+1}}.
  \end{align*}
  Using the expressions for the scheme and Taylor expanding in
  $\Delta t$ we get
  \begin{align*}
    \tan\theta^{n+1}
    =
    \tan\theta^n +
    \frac{\omega E_0^2}{(v^{n})^2}\Delta t +
    \frac{\omega^3 z^n E_0^2}{(v^n)^3}\Delta t^2
    + O(\Delta t^3)
  \end{align*}
  The first three terms match the Taylor expansion of the exact
  solution $\tan(\theta^n+\omega\Delta t)$ and the last term is the
  \emph{phase-error}.
\end{frame}

\begin{frame}{Single particle motion in an electromagnetic field}
  \small%
  \begin{itemize}
  \item In PIC method the Vlasov-Maxwell equation is solved in the
    \emph{Lagrangian frame}: the phase-space is represented by
    \emph{finite-sized} ``macro-particles''.
  \item In the Lagrangian frame the distribution
    function remains constants along \emph{characteristics} in
    phase-space.
  \item These characteristics satisfy the ODE of particles moving
    under Lorentz force law
    \begin{align*}
      \frac{d\mvec{x}}{dt} &= \mvec{v} \\
      \frac{d\mvec{v}}{dt} &= \frac{q}{m}(\mvec{E}(\mvec{x},t) + \mvec{v}\times\mvec{B}(\mvec{x},t))
    \end{align*}
  \item In the absence of an electric field, the kinetic energy must
    be conserved
    \begin{align*}
      \frac{1}{2} |\mvec{v}|^2 = \textrm{constant}.
    \end{align*}
    This is independent of the spatial or time dependence of the
    magnetic field. Geometrically this means that in the absence of an
    electric field the velocity vector rotates and its tip always lies
    on a sphere.
  \end{itemize}
\end{frame}

\begin{frame}{Single particle motion in an electromagnetic field}
  \small%
  \begin{itemize}
  \item A mid-point scheme for this equation system would look like
    \begin{align*}
    \frac{\mvec{x}^{n+1}-\mvec{x}^n}{\Delta t} &= \frac{\mvec{v}^{n+1}+\mvec{v}^n}{2} \\
    \frac{\mvec{v}^{n+1}-\mvec{v}^n}{\Delta t} &= \frac{q}{m}
      \big(
        \overline{\mvec{E}}(\mvec{x},t) +
        \frac{\mvec{v}^{n+1}+\mvec{v}^n}{2}\times\overline{\mvec{B}}(\mvec{x},t)
      \big)
    \end{align*}
    The overbars indicate some averaged electric and magnetic fields
    evaluated from the new and old positions. In general, this would
    make the scheme nonlinear!

  \item Instead, we will use a \emph{staggered} scheme in which the
    position and velocity are staggered by half a time-step.
    \begin{align*}
    \frac{\mvec{x}^{n+1}-\mvec{x}^n}{\Delta t} &= \mvec{v}^{n+1/2} \\
    \frac{\mvec{v}^{n+1/2}-\mvec{v}^{n-1/2}}{\Delta t} &= \frac{q}{m}
      \big(
        \mvec{E}(\mvec{x}^n,t^n) +
        \frac{\mvec{v}^{n+1/2}+\mvec{v}^{n-1/2}}{2}\times\mvec{B}(\mvec{x}^n,t^n)
      \big)
    \end{align*}
  \end{itemize}
\end{frame}

\begin{frame}{The Boris algorithm for the staggered scheme}
  \footnotesize%
  The velocity update formula is
  \begin{align*}
    \frac{\mvec{v}^{n+1/2}-\mvec{v}^{n-1/2}}{\Delta t} &= \frac{q}{m}
      \big(
        \mvec{E}(\mvec{x}^n,t^n) +
        \frac{\mvec{v}^{n+1/2}+\mvec{v}^{n-1/2}}{2}\times\mvec{B}(\mvec{x}^n,t^n)
      \big)
  \end{align*}
  This appears like an implicit method: most obvious is to construct a
  linear $3\times 3$ system of equations and invert them to determine
  $\mvec{v}^{n+1}$. Puzzle to test your vector-identity foo: find
  $\mvec{A}$ if $\mvec{A} = \mvec{R}+\mvec{A}\times\mvec{B}$.

  The Boris algorithm updates this equation in three steps:
  \begin{align*}
    \mvec{v}^- &= \mvec{v}^{n-1/2} + \frac{q}{m}\mvec{E}^n\frac{\Delta t}{2} \\
    \frac{\mvec{v}^+- \mvec{v}^-}{\Delta t}
    &=
      \frac{q}{2 m}(\mvec{v}^++\mvec{v}^-)\times\mvec{B}^n \\
    \mvec{v}^{n+1/2}
    &=
      \mvec{v}^{+} + \frac{q}{m}\mvec{E}^n\frac{\Delta t}{2} \\    
  \end{align*}
  Convince yourself that this is indeed equivalent to the staggered
  expression above. So we have two electric field updates with half
  time-steps and a rotation due to the magnetic field. Once the
  updated velocity is computed, we can trivially compute the updated
  positions.
\end{frame}  

\end{document}


\begin{frame}{}
\end{frame}

% ----------------------------------------------------------------
