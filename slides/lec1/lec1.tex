\documentclass[pdf]{beamer}
\usepackage{hyperref}
%% AMS packages and font files
\usepackage{amsmath}
\usepackage{amsfonts}
\usepackage{amsthm}
\usepackage{stmaryrd}
\usepackage{caption}

\captionsetup{font=scriptsize,labelfont=scriptsize}
\footnotesize\small

\DeclareMathAlphabet{\mathpzc}{OT1}{pzc}{m}{it}

\newcommand{\eqr}[1]{Eq.\thinspace(#1)}
\newcommand{\pfrac}[2]{\frac{\partial #1}{\partial #2}}
\newcommand{\pfracc}[2]{\frac{\partial^2 #1}{\partial #2^2}}
\newcommand{\pfraca}[1]{\frac{\partial}{\partial #1}}
\newcommand{\pfracb}[2]{\partial #1/\partial #2}
\newcommand{\pfracbb}[2]{\partial^2 #1/\partial #2^2}
\newcommand{\spfrac}[2]{{\partial_{#1}} {#2}}
\newcommand{\mvec}[1]{\mathbf{#1}}
\newcommand{\gvec}[1]{\boldsymbol{#1}}
\newcommand{\script}[1]{\mathpzc{#1}}
\newcommand{\eep}{\mvec{e}_\phi}
\newcommand{\eer}{\mvec{e}_r}
\newcommand{\eez}{\mvec{e}_z}
\newcommand{\iprod}[2]{\langle{#1}\rangle_{#2}}
%\newcommand{\mypause}{\pause}
\newcommand{\mypause}{}
\newcommand{\gcs}{\nabla_{\mvec{x}}}
\newcommand{\gvs}{\nabla_{\mvec{v}}}
\newcommand{\gps}{\nabla_{\mvec{z}}}
\newcommand{\dtv}{\thinspace d^3\mvec{v}}
\newcommand{\dtx}{\thinspace d^3\mvec{x}}

\newtheorem{thm}{Theorem}
\newtheorem{lem}{Lemma}
\newtheorem{remark}{Remark}

\newtheorem{proposition}{Proposition}

\theoremstyle{definition}
\newtheorem{dfn}{Definition}
%\newtheorem{thm}{Theorem}
\newtheorem{proposed}{Proposed Answer}
\DeclareMathOperator{\spn}{span}

%% autoscaled figures
\newcommand{\incfig}{\centering\includegraphics}
\setkeys{Gin}{width=0.7\linewidth,keepaspectratio}

\usepackage{beamerthemesplit}
\setbeamercovered{transparent}
%\usetheme{lined}
%\usetheme{CambridgeUS}
%\usecolortheme{dolphin}
\usetheme{Madrid}
\usecolortheme{beaver}

\usefonttheme[onlymath]{serif}
\setbeamertemplate{navigation symbols}{}

% Setup TikZ
\usepackage{tikz}
\usetikzlibrary{arrows}
\tikzstyle{block}=[draw opacity=0.7,line width=1.4cm]

\title[{\tt }] {Computational Methods in Plasma Physics. Lecture I}%
\author[http://cmpp.rtfd.io]%
{Ammar H. Hakim\inst{1}}%

\institute[PPPL]
{ \inst{1} Princeton Plasma Physics Laboratory, Princeton, NJ %
}

\date[8/12/2019]{PPPL Graduate Summer School, 2019}

\begin{document}

\begin{frame}
  \titlepage
\end{frame}

\begin{frame}{Goal: Modern Computational Techniques for Plasma Physics}
  \footnotesize Vast majority of plasma physics in contained in the
  Vlasov-Maxwell equations that describe self-consistent evolution of
  distribution function $f(\mvec{x},\mvec{v},t)$ and electromagnetic
  fields:
  \begin{align*}
    \pfrac{f_s}{t} + \gcs\cdot (\mvec{v} f_s) + \gvs\cdot (\mvec{F}_s
    f_s) = \left( \pfrac{f_s}{t} \right)_c
  \end{align*}
  where $\mvec{F}_s=q_s/m_s (\mvec{E}+\mvec{v}\times\mvec{B})$. The EM
  fields are determined from Maxwell equations
  \begin{align*}
    \frac{\partial \mvec{B}}{\partial t} + \nabla\times\mvec{E} &= 0 \\
    \epsilon_0\mu_0\frac{\partial \mvec{E}}{\partial t} -
    \nabla\times\mvec{B} &= -\mu_0\sum_s q_s \int_{-\infty}^\infty v f_s \thinspace d\mvec{v}^3
  \end{align*}
  Highly nonlinear: fields tell particles how to move. Particle motion
  generates fields. This is a very difficult system of equations to
  solve! Theoretical and computational plasma physics consists of
  making approximations and solving these equations in specific
  situations.
\end{frame}

\begin{frame}{Why is solving Vlasov-Maxwell equations directly so
    hard?}
  Despite being the fundamental equation in plasma physics the VM
  equations remain highly challenging to solve.
  \begin{itemize}
  \item Highly nonlinear with the coupling between fields and
    particles via currents and Lorentz force. Collisions can further
    complicate things due to long-range forces in a plasma; dominated
    by small-angle collisions
  \item High dimensionality and multiple species with large mass
    ratios: 6D phase-space, $m_e/m_p = 1/1836$ and possibly dozens of
    species.
  \item Enormous scales in the system: light speed and electron plasma
    oscillations; cyclotron motion of electrons and ions; fluid-like
    evolution on intermediate scales; resistive slow evolution of
    near-equilibrium states; transport scale evolution in tokamak
    discharges.  14 orders of magnitude of physics in these
    equations!
  \end{itemize}
\end{frame}

\begin{frame}{Many approximations developed over the decades}
  \small%
  Modern computational plasma physics consists of making justified
  approximations to the VM system and then coming up with efficient
  schemes to solve them.
  \begin{itemize}
  \item Major recent theoretical development in plasma physics is the
    discovery of gyrokinetic equations, an asymptotic approximation
    for plasmas in strong magnetic fields. Reduces dimensionality to
    5D (from 6D) and eliminates cyclotron frequency and gyroradius
    from the system. Very active area of research.
  \item Many fluid approximations have been developed to treat plasma
    via low-order moments: extended MHD models; multimoment models;
    various reduced MHD equations
  \item Numerical methods for these equations have undergone
    renaissance in recent years: emphasis on \emph{memetic} schemes
    that preserve conservation laws and some geometric features of the
    continuous equations. Based on Lagrangian and Hamiltonian
    formulation of basic equations. Very active area of research.
  \end{itemize}
  With advent of large scale computing much research is now focused on
  schemes that scale well on thousands (millions) of CPU/GPU cores.
\end{frame}

\begin{frame}{Goal of this course is to look at some key schemes and
    study their properties}
  \small%
  Computational plasma physics is vast: we can only cover a (very)
  small fraction of interesting methods. In this class we will focus
  on
  \begin{itemize}
  \item Solving the Vlasov-Maxwell equations using particles, the
    ``Particle-in-cell'' method; methods to solve Maxwell equations
    (Lecture 1 and 2). This is probably the most widely used method
    that yields reasonable results for many kinetic problems
  \item Shock-capturing methods for plasma fluid equations. These are
    particularly relevant to astrophysical problems in which flows can
    be supersonic or super-Alfvenic. A brief look at fluid solvers for
    use in fusion machines (tokamaks, stellarators) in which dynamics
    is much slower. (Lecture 3)
  \item Directly discretizing the Vlasov-Maxwell equations as a PDE in
    6D. This is an emerging area of active research and may open up
    study of turbulence in fusion machines and also exploring
    fundamental plasma physics in phase-space. (Lecture 4)
  \end{itemize}
\end{frame}

\begin{frame}{Conservation properties of Vlasov-Maxwell equations}
  \small%
  It is important to design methods that preserve at least some
  properties of continuous Vlasov-Maxwell system. Define the moment
  operator for any function $\varphi(\mvec{v})$ as
  \begin{align*}
    \langle \varphi(\mvec{v}) \rangle_s
    \equiv
    \int_{-\infty}^{\infty} \varphi(\mvec{v}) f_s(t,\mvec{x},\mvec{v}) \dtv.
    \label{eq:momentDefinition}
  \end{align*}
  The Vlasov-Maxwell system conserves particles:
  \begin{align*}
    \frac{d}{dt}\int_\Omega \sum_s \langle 1 \rangle_s \dtx = 0.
  \end{align*}
  The Vlasov-Maxwell system conserves total (particles plus field)
  momentum.
  \begin{align*}
    \frac{d}{dt} \int_\Omega \left( \sum_s \langle m_s\mvec{v} \rangle_s + \epsilon_0 \mvec{E}\times\mvec{B} \right) \dtx = 0.
  \end{align*}
  The Vlasov-Maxwell system conserves total (particles plus field) energy.
  \begin{align*}
    \frac{d}{dt} \int_\Omega \left(\sum_s \langle \frac{1}{2} m_s |\mvec{v}|^2 \rangle_s + \frac{\epsilon_0}{2} |\mvec{E}|^2 + \frac{1}{2\mu_0} |\mvec{B}|^2 \right) \dtx = 0.
  \end{align*}
\end{frame}

\begin{frame}{Conservation properties of Vlasov-Maxwell equations}
  \small
  Besides the fundamental conservation laws, in the \emph{absence of
    collisions} we can also show that
  \begin{align*}
    \frac{d}{dt}\int_K \frac{1}{2} f_s^2\thinspace d\mvec{z} = 0,
  \end{align*}
  where the integration is taken over the complete phase-space. Also,
  the entropy is a \emph{non-decreasing} function of time
  \begin{align*}
    \frac{d}{dt}\int_K -f_s \ln(f_s) \thinspace d\mvec{z} \ge 0.
  \end{align*}
  For collisionless system the entropy remains \emph{constant}.
  (Prove these properties as homework/classwork problems).
  \begin{itemize}
  \item It is not always possible to ensure all these properties are
    preserved numerically. For example: usually one can either ensure
    momentum \emph{or} energy conservation but not both; it is very
    hard to ensure $f(t,\mvec{x},\mvec{v})>0$.
  \item Much of modern computational plasma physics research is aimed
    towards constructing schemes that preserve these properties.
  \end{itemize}
\end{frame}

\begin{frame}{Single particle motion in an electromagnetic field}
  \small%
  \begin{itemize}
  \item In the \emph{Particle-in-cell} (PIC) method the Vlasov-Maxwell
    equation is solved in the \emph{Lagrangian frame} in which the
    phase-space is represented by \emph{finite-sized}
    ``macro-particles''.
  \item In the Lagrangian frame the distribution
    function remains constants along \emph{characteristics} in
    phase-space.
  \item These characteristics satisfy the ODE of particles moving
    under Lorentz force law
    \begin{align*}
      \frac{d\mvec{x}}{dt} &= \mvec{v} \\
      \frac{d\mvec{v}}{dt} &= \frac{q}{m}(\mvec{E}(\mvec{x},t) + \mvec{v}\times\mvec{B}(\mvec{x},t))
    \end{align*}
  \item We will first focus on solving the equations-of-motion for
    the macro-particles, leaving solution of Maxwell equations and
    coupling to particles for Lecture 2.
  \end{itemize}
\end{frame}

\begin{frame}{Simple harmonic oscillator}
  \small%
  Consider first the simple harmonic oscillator
  \begin{align*}
    \frac{d^2z}{dt^2} = -\omega^2 z^2
  \end{align*}
  This has exact solution $z = a\cos(\omega t) + b\sin(\omega t)$,
  where $a$ and $b$ are arbitrary constants. How to solve this
  numerically? Write as a system of first-order ODEs
  \begin{align*}
    \frac{dz}{dt} = v; \quad \frac{dv}{dt} = -\omega^2 z
  \end{align*}
  Note that the coordinates $(z,v)$ label the \emph{phase-space} of
  the harmonic oscillator. Multiply the second equation by $v$ and use
  the first equation to get
  \begin{align*}
    \frac{d}{dt}\left(\frac{1}{2} v^2 + \frac{1}{2}\omega^2 z^2\right) = 0.
  \end{align*}
  This is the \emph{energy} and is \emph{conserved}.%
  
  \bf{Question: how to solve the ODE such that the energy is conserved
    by the \emph{discrete scheme}?}
\end{frame}

\begin{frame}{Harmonic oscillator: Forward Euler Scheme}
  First attempt: use the simplest possible scheme, replace derivatives
  with difference approximations
  \begin{align*}
    \frac{z^{n+1}-z^n}{\Delta t} = v^n; \quad \frac{v^{n+1}-v^n}{\Delta t} = -\omega^2 z^n
  \end{align*}
  or
  \begin{align*}
    z^{n+1} = z^n + \Delta t v^n; \quad v^{n+1} = v^n - \Delta t \omega^2 z^n
  \end{align*}
  This is the \emph{forward Euler} scheme. Lets check if the discrete
  scheme conserves energy:
  \begin{align*}
    (v^{n+1})^2 + \omega^2 (z^{n+1})^2 =
    (1+\omega^2 \Delta t^2)((v^{n})^2 + \omega^2 (z^{n})^2)
  \end{align*}
  The presence of the $\omega^2 \Delta t^2$ in the bracket spoils the
  conservation. So the forward Euler scheme \emph{does not} conserve
  energy. Also, note that the energy, in fact, is \emph{increasing}!
\end{frame}

\begin{frame}{Harmonic oscillator: Forward Euler Scheme}
  Closer look: write as a matrix equation
  \begin{align*}
    \left[
    \begin{matrix}
      z^{n+1} \\
      v^{n+1}
    \end{matrix}
    \right]
    =
    \underbrace{
    \left[
    \begin{matrix}
      1 & \Delta t \\
      -\omega^2 \Delta t & 1
    \end{matrix}
    \right]
    }_{\textrm{Jacobian},\ J}
    \left[
    \begin{matrix}
      z^{n} \\
      v^{n}
    \end{matrix}
    \right].                        
  \end{align*}
  Observe that the determinant of the Jacobian is
  $\det(J)=(1+\omega^2\Delta t^2)$ which is the same factor as appears
  in the energy relation. One may reasonably conjecture that when this
  determinant is one, then perhaps energy is conserved.

  \begin{block}{Volume Preserving Scheme}
    We will call say a scheme preserves \emph{phase-space} volume
    if the determinant of the Jacobian is $\det(J)=1$.
  \end{block}

\end{frame}

\begin{frame}{Harmonic oscillator: Mid-point Scheme}
  \small%
  Perhaps a better approximation will be obtained if we use
  \emph{averaged} values of $z,v$ on the RHS of the discrete
  equation:
  \begin{align*}
    \frac{z^{n+1}-z^n}{\Delta t} &= \frac{v^n +v^{n+1}}{2} \\
    \frac{v^{n+1}-v^n}{\Delta t} &= -\omega^2\frac{z^n + z^{n+1}}{2}
  \end{align*}
  This is an \emph{implicit} method as the solution at the next
  time-step depends on the old as well as the next time-step
  values. In this simple case we can explicitly write the update in a
  matrix form as
  \begin{align*}
    \left[
    \begin{matrix}
      z^{n+1} \\
      v^{n+1}
    \end{matrix}
      \right]
      =
      \frac{1}{1+\omega^2\Delta t^2/4}
    \left[
      \begin{matrix}
        1-\omega^2\Delta t^2/4  & \Delta t \\
        -\omega^2 \Delta t & 1-\omega^2\Delta t^2/4
      \end{matrix}
    \right]
    \left[
    \begin{matrix}
      z^{n} \\
      v^{n}
    \end{matrix}
    \right].                        
  \end{align*}
  For this scheme $\det(J)=1$. So the mid-point scheme conserves
  phase-space volume! Some tedious algebra also shows that
  \begin{align*}
    (v^{n+1})^2 + \omega^2 (z^{n+1})^2 = (v^{n})^2 + \omega^2 (z^{n})^2
  \end{align*}
  showing that energy is also conserved by the mid-point scheme!
\end{frame}

\begin{frame}{Harmonic oscillator: Mid-point Scheme is symplectic}
  A more stringent constraint on a scheme for the simple harmonic
  oscillator is that it be \emph{symplectic}. To check if a scheme is
  symplectic one checks to see if
  \begin{align*}
    J^T\sigma J = \sigma
  \end{align*}
  where $\sigma$ is the \emph{unit symplectic matrix}
  \begin{align*}
    \sigma
    =
    \left[
      \begin{matrix}
        0 & 1 \\
        -1 & 0
      \end{matrix}
    \right]    
  \end{align*}
  Turns out that the mid-point scheme for the harmonic oscillator is
  also symplectic. Note that if a scheme conserves phase-space
  volume, it \emph{need not} be symplectic.
\end{frame}  

\end{document}


\begin{frame}{}
\end{frame}

% ----------------------------------------------------------------
